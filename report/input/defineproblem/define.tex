
As defined in \secref{sec:problemDef} this project is concerned with integrating the Aalborg PBL model into Moodle.
An important concept of this model is the ``Team'' (see \secref{sub:aaupbl}).
For a team to operate in an interactive environment it is necessary to have some entity that defines the members of the team and the actions that they can perform to solve their problem.
We call this entity a ``Project Group''.

To make \system{} usable at any educational institution we provide some form of administrative tool to manage project groups -- e.g. allowing  creation, modification, and archiving of project groups.
The members of the project group will have access to some shared activities through which they can conduct they group work.
These activities include planning the course of the project, communicating internally during the course of the project, and communicating with people not actively part of the project group, such as supervisors (se mentioned \secref{sub:decomposingSys}).
These activities are created in our peer sub-projects.
Our responsibility is to make all these activities available to the members of the project groups.
\section{System Definition}
\label{sec:subSysDef}
From the initial analysis and the intro to this chapter we define the subsystem we are implementing as: \todo{find  what exactly this shoulkd reference}
\begin{center}
\framebox[0.85\textwidth][c]{
	\parbox{0.8\textwidth}{
		\textsl{
			A subsystem of \system{} that implements project groups in Moodle and allow for administration and usage thereof. 
			The subsystem includes a virtual meeting place, which integrates the other subsystems. 
		}
	}
}
\end{center}
The subsystem must conform to the overall system definition, defined in~\secref{sec:systemDef}, and may not interfere with existing Moodle functionality.	