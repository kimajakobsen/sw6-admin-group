\section{End Users}
\label{sec:enduser}
An end user is a person that will be using the final system.
Since we are using scrum as our primary development practice we would prefer to have a Product Owner~\cite[p.~115]{Larman04} that can speak on behalf of the end users.
In this section we will define the group of end users for our \subsystem{} and shortly present the people that we use as representatives for this group -- we cannot speak with the entire group of potential end users, because it is simply too large.

For the full system the group of end users (also referred to as target group) is very diverse.
Roughly divided, it contains students, supervisors, and \admpers{} such as secretaries.
If we continue the division of students there are many dimensions to consider. 
For instance different students have different levels of experience with LMSs such as Moodle. Another dimension is that students have study at different faculties.
%The granularity may be even finer by partitioning by time of Moodle experience, such years of Moodle usage.
These different dimensions also apples to supervisors and \admpers{}.
We regard our target group as divided into two categories, users that are members of project groups and users that use the administrative tool to manage project groups.
In \secref{sub:endusersmembers} and \secref{sub:enduserstool} we present the characteristics of our target group.\todo{kontorller at det er de rigtige personer der bliver talt om i de næste sectioner.}
%A table showing an overview of these two groups of end users is seen in \figref{fig:endusers}.

\subsection{Members of Project Groups}
\label{sub:endusersmembers}
As discussed in \chapref{chap:systemDef} our responsibility is to create the concept of a project group and creating a tool to manage these groups.
%For the concept of project group a group of students will function as our target group.
Since we are working together with three other peer-groups we will receive requests for functionality from them.
Our peer-groups have their own end users, both students and supervisors, where they get their requests from.
We can choose to include students in our target group as well and make our own field studies.
Alternatively we can choose to rely on our peer-groups requests and effectively use our peer-groups as our target group.
We choose to go a middle way and rely on our peer-groups requests and cooperate in the field studies related to the entire system or the concept of project groups.

The end users of project group functionalities are students and supervisors.
We want to have student representatives of different types, in particular students from different faculties.
We have two students from The Faculty of Social Sciences and one from The Faculty of Natural Sciences.
These are respectively Katrine Holmgaard Dinitzen, Mathilde Gammelgaard, and Lea Gustafsson.
Regarding supervisors, we choose to have a single representative; Thomas Ryberg Vibjerg Hansen, who is supervisor under The Faculty of Humanities.
%We focus on students since our peer-group \supervisorgroup{}

\subsection{Users of Administrative Tool}
\label{sub:enduserstool}
Regarding the administrative tool, we choose to use administrative personnel as our main target group.
The end users we are using come from Aalborg University.
The reason for this is two-fold.
First reason is that we are implementing the Aalborg PBL model into Moodle, which means that Aalborg University come as a natural choice to look for end users.
Secondly our project is being conducted at Aalborg University, consequently it is easy for us to find personnel to participate.
To get a diverse group of end user we contact administrative personnel from our own institute and two other institutions at Aalborg University.
We also want our group of end user to contain both personnel well experienced in Moodle and other LMSs, and personnel inexperienced in Moodle.
From our own institute, Institute of Computer Science, we have the representatives Lene Winther Even, Camilla G\ae{}raa Larsen, and Mikael M\o{}ller Hansen.
Where the two former are a senior secretary and an office trainee respectively, and the latter an IT administrator
From The Faculty of Social Sciences we have been in contact with  Jette Due Nielsen and Pia Knudsen as our representatives.
We have also been in contact with Morten Mathiasen Andersen from MPBL in addition to the initial discussion discussed in \secref{sub:mpblInterview}.
