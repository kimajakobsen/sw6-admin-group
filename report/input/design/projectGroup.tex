\section{Project Group}
\label{sec:projectgroup}
In this section the different aspects of the concept of the project groups will be described.
\subsection{Project Group Room}
A project group room is a page that serves as a virtual group room. 
It should have some of the same functionality as a physical group room.
We as the \groupname{} implement the page, and other three groups implement the blocks (see \secref{subsec:blocks}) that serve as different functionality.
A good aspect of using blocks is that it is easy to manage which blocks should be shown and where they should be shown on the page.
The project group room is a place where students can exchange information among themselves and with their supervisor.

\todo{Nogen der kan komme i tanke om noget mere at skrive her?}

\subsection{Navigation}
There is currently a problem in Moodle when it comes to navigation. 
When a user is enrolled in a large number of courses their entire front page is filled with links to those courses.
There is no built-in functionality to move or sort the links.
ELSA mentioned this problem during the meeting that we conducted with them described in \secref{sub:elsaInterview}.
%ref Lene
\todo{insert refs to interview with Lene}

We want to avoid this problem when we design the navigation for project groups.
Moodle has a navigation block with a list of important links.
We want to add an item to this list that, when expanded, shows the project groups the user is a member of.
Since a supervisor or an administrator might be a member of a many project groups, we want to limit the size of the list of project groups.
We do this by showing at max a set value, and if the user is a member of more groups we show a link to a page that has a list of all the users's group.

\subsection{Managing Project Groups}
\label{sec:s}
Administrators need to be able to add, remove, and edit project groups.
Per default Moodle has a block called Settings. 
If a user is an administrator there is a list in this block called Site administration. 
This list contains a lot of tools administrators use, and it is only natural to add an item to this list, where administrators can access project group functionality.

We want a page where administrators can add a new project group that also adheres to Moodle standards.
It should be able to name a group, add the desired members, and choose who the supervisor(s) of the group is/are.

Additionally there needs to be a list of all project groups.
This list can potentiality grow very large, so there needs to a way to find a specific group or a specific set of groups. \todo{lene interview ref}
From this list it should be possible to add, edit, and delete groups. 

All in all this functionality should be implemented to be as efficient as possible, while still being user friendly.
Usability is not the biggest concern since we expect administrators to be expert users.
This means that they will be using the system often and once functionality is learned by a user it is unlikely that they will forget again.