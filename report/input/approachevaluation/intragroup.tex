\section{Intra Group Development}
\label{sec:intragroupdev}
%what will this section be about.

\subsection{Scrum}
%The development method we used in this project is \scrum{}.
We were unable to follow the scrum development method completely because some roles were not filled.
While still a problem, fulfilling all roles is not paramount in the intra group development of this project.
We did not have a constant \scrum{} master, instead we took turns taking the role of \scrum{} master.
This worked sufficiently well, but it would be preferable to have the role filled by an experienced \scrum{} master.
One could argue that it would have been preferable to have only one person take the role of \scrum{} master, since he would gain more experience and the development would be more uniform.
However, since this is a learning project we all would benefit from being the \scrum{} master.

An issue in the implementation of \sos{} is the lack of a product owner, which is described as key concept. 
We compensated for this lack by doing several interviews with possible end users and to have them included in the demo meetings. 
In effect we have had the role of the product owner our selves, which is not an optimal solution. 
%It is not optimal for the reason that it is recognized as bad \scrum{} practice not to have a product owner to prioritize tasks \cite[p.~128]{Larman04}. 
It is not optimal for the reason that it is recognized as bad \scrum{} practice to have many product owners to prioritize and select tasks for a sprint~\cite[p.~128]{Larman04}. 
Furthermore, we have conflicting interests -- we want to implement something exciting and challenging, while the end users may in fact need something we consider to be simple and boring.

%skriv om hvordan vores planning poker har udviklet sig. 
%skriv om hvordan story points are udviklet sig. og hvordan vi har lært at bruge dem. FYI så er story points ikke forklaet i rapporten, men kun nævnt.


\subsection{Task Management}
When planning a sprint we used planning poker, which is a \scrum{} practice to estimate how much time a task takes. 
%This method got progressively better, because
In the beginning we had very different ideas of how much time a story point was and we did not decide anything specific. 
After working on tasks our valuation of story points became increasingly similar over time.
%backlog
%The backlog items we valuated by planning poker were split into smaller tasks as work began on them.

%% Taksskalduha
We tried several different approaches for defining tasks and backlog items. 
In the first sprint we created a few backlog items and after estimating them using planing poker we decomposed the items into smaller tasks, which then got evaluated. 
This lead to a large amount of small tasks, but as progress went on we realized that some tasks were redundant and could be combined, while others were missing to fulfill the backlog items requirements. 

In the second sprint we tried to estimate only the backlog items and not create smaller tasks. 
This made it difficult to hand out small tasks and evaluation the progress during the sprint. 

In the third sprint we created one task for each backlog item and the person who took the tasks had the job of starting to program and create smaller tasks that other people can take. 
This proved to be the most efficient method of distributing tasks.

A burndown chart of remaining story points was a part of the tasks management.
It was positioned in a visible position and served as a motivational tool.
%The burndown chart used as part of the development was located in a visible position and worked well as motivational tool.


\subsection{End Users}
During the development we relied on several end users in the gathering of our requirements.
The end users selected was not as spread out as we initially had wished for. 
This is mainly because it is difficult to find end users who are willing to volunteer for interviews. 
For both target groups, the users of the project groups and the administrative personnel, we lack end users from the Faculty of Medicine.
For the usage of the virtual group room we need users with high experience and more than one supervisor for interviewing. 
The fact that we did not fully cover all types of people for interviews does not render the requirements useless, but it gives room for further improvement.