\section{Virtual Meeting Place}
\label{sec:virtualMeetingPlace}

%%Skriv om hvad vi vil gøre nu . først fide information så finde besluitnings mligheder
We will now account for the different possible manners in which we can design the virtual meeting place. We will do this by first listing the criteria that the different options are based on. First we consider if and how the the virtual group room should be design on a high level. We will then go through the levels listing the different options. The final design will be decided upon in the \nameref{chap:design} chapter. 

Our problems statement states, we are to develop a virtual meeting place. 
The notion of a virtual meeting place originates from two sources.

\paragraph{Aalborg PBL} 
At Aalborg University project groups are often assigned to group rooms. This relation exists in one of three ways:
\begin{itemize}
	\item \textbf{No Group Rooms} There are no group available because the field of study dose not follow the Aalborg PBL model.
	\item \textbf{Daily} The students have reserve the group rooms on a daily basis.
	\item \textbf{Half-yearly} Group rooms are assigned to each project group at the start of a semester.
\end{itemize} 
The physical group room can play different roles. We assume that the optimal relation is group rooms on a half-yearly basis.
The concept of a group room can be extended to a virtual meeting place.  \todo{Mangler kilde}  

\paragraph{MPBL nearness}
In the interview with MPBL described in \secref{}, they suggested that to create a feeling of nearness, a virtual meeting place could be created. This should to serve as a gathering point where cooperation tools should be available. \\

The virtual meeting place should be a but where all central tools are available. It should be similar to the Aalborg group room concept and be available on a half-yearly basis.  

We know that other LMS exist beside from Moodle from \secref{sec:LMS}. 
SharePoint have course groups that allow for cooperation in regards to a course. Litmos introduce groups with a recursive tree structure. SharePoint and Litmos represents two different ways to design virtual project groups. Moodles native group concept, described in \secref{sec:groups}, have its groups linked to courses just like SharePoint.   \todo{skriv også hvordan moodle gør. skriv det her}\\


\todo{Maaske skriv noget om nogle flere criterias}
%%Cirteria - what is important for us
%Problem orientation
%Project organization 
%Integration of theory and practice
%Participant direction 
%Team-based approach 
%Collaboration and feedback 


We will now consider how the different options of how to design the virtual group room.

\subsection{Virtual group room}
The members of a project group should have a place where they can meet and engage in project related activities.
Recall from \secref{sec:subSysDef} that our responsibility is to construct a place where this can happen, not construct the actual activities -- that is the responsibility of our peer-groups.
We have to decide the scope of the virtual meeting place.

\paragraph{Shared Group} One idea is to have every project group share everything with other project groups in a semester.
This corresponds to having every project group working in the same room in the real world.

\paragraph{Team Group} Another idea is to give a virtual group room to every project group and ensuring that only the members of the project group and the supervisors can contribute to the work on the project.
The corresponding situation in the real world is that every project group has their own group room where only they (and there supervisors) can do work on the project. \\

The Shared Group idea is easier than Team Group to implement since no permissions need to be considered.
However, the Team Group is closer to the way that Aalborg PBL is implemented -- with a team working together on the project as described in \secref{sub:aaupbl}.

We choose the Team Group idea. 
It is infeasible to have every project group share everything.
A project group member could be forced to look through many functionalities to find the one relevant to the given project group.
Furthermore, by allowing each project group to have there own virtual group room, the members can customize the place as they see fit by removing irrelevant functionalities and possibly adding new relevant functionalities.
For the rest of the report we refer to the virtual group room simply as the ``project group room''.

The project group room should have some of the same functionality as a physical group room.
We as the \groupname{} implement the project group room, and the other three peer-groups implement the blocks (see \secref{subsec:blocks}) that provide as different functionality.
%wrapper for the three other groups
%Joining of tools

Now that the fundamental design is decided, are we able to consider the different types of project group rooms and their properties.

\subsection{Project Group Room}
In the Aalborg PBL model project and teams exits as two different entities. The question arises if project groups should be design as groups that have a relation to a project or as an single entities where the group and project are one. We will now list the properties of the two different variants. 

\paragraph{Groups and Projects divided} We now consider the properties of Groups and Projects as two different entities:
\begin{itemize}
	\item Projects and Groups have a many-to-many relationship.
	\item Directly reflects the objects of Aalborg PBL model.
	\item Complex to implement.
\end{itemize}


\paragraph{Groups and Projects as one} We now consider the properties of Groups and Projects as one single entity:
\begin{itemize}
	\item A group have a project, they are inseparable.
	\item Simple to implement.
\end{itemize}


We will now examine related design options.

\subsection{Recursive tree structure}
Regardless of the group and project relation the structure of project group room rather simple. 
Litmos's groups have a recursive tree structure. Project groups can be extended such that a project group can consist of other groups. An example of this is at the Software Engineering 6\nth{} semester. The students are divided into two main projects. Each of these contains project groups. 







%%%design

%What attribtes should it have
%Aalborg university does not distinguesh between project and groups \ref{sec:lene}