\section{Intra Group Development}
\label{sec:intragroupdev}
The development method we used in this project is \scrum{}.
We were unable to follow the scrum development method completely because some roles were not filled.
While still a problem fulfilling all roles is not paramount in the intra group development of this project.
We did not have a constant \scrum{} master, instead we took turns taking the role of \scrum{} master.
This worked sufficiently well, but it would be preferable to have the roll filled by an experienced \scrum{} master.
One could argue that it would have been preferable to only have one person take the role of \scrum{} master, since he would gain more experience and the development would be more uniform.
However, since this is a learning project we all would benefit from being the \scrum{} master.

%skriv om hvordan vores planning poker har udviklet sig. 
%skriv om hvordan story points are udviklet sig. og hvordan vi har lært at bruge dem. FYI så er story points ikke forklaet i rapporten, men kun nævnt.
When planning a sprint we used planning poker, which is a \scrum{} practice.
This method got progressively better, because our valuation of story points became increasingly similar over time.

%backlog
%The backlog items we valuated by planning poker were split into smaller tasks as work began on them.


%burndown chart

