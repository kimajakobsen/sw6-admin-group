\documentclass[a4paper]{article}


% tilføjet af Stubman 231010
\usepackage[footnote,draft,danish,silent,nomargin]{fixme}		% Indst rettelser og lignende med \fixme{...} Med final i stedet for draft, udlses en error 																															for hver fixme, der ikke er slettet, nr rapporten bygges.

%%%%%%%%%% Various Packages %%%%%%%%%
\usepackage[english]{babel}
\usepackage[utf8]{inputenc}
\usepackage[numbers]{natbib}
\usepackage{verbatim}%% For Comment enviroment
%\usepackage{listings}

\usepackage{lastpage}% gives lastpage commando
\usepackage{algorithm}%%% Algorithm Eviroment
\usepackage{algorithmic}%%% Algorithm Eviroment
\usepackage{amsmath, amsfonts, amssymb, amsthm,mathtools} % Math Paths
\usepackage{bm}
\usepackage{fancyhdr} % For headers
%\usepackage{sadlist} % Selfdefined package. Gives \begin{sadlist}{title}{description}{label}
%\usepackage{casecontrol}%
%\usepackage{packages/maplestd2e}%
%\usepackage{xifthen}% provides \isempty test and ifthen else 
\usepackage[absolute]{textpos} %used on the frontpage for the picture.
\usepackage{tabularx}
%\usepackage[retainorgcmds]{IEEEtrantools}
%\usepackage{qtree}
%\usepackage{stmaryrd}
%\usepackage{morefloats}
%\usepackage{placeins} % Gives the \FloatBarrier command
%\usepackage{pdfpages}
%\usepackage{rotating}

%%%%%%%%%%%% Words %%%%%%%%%%%%%%
\newcommand{\elearning}[1][]{\caseControl{e}{-learning}{#1}{}}%



%%%%%%%%%%%%%%%%%%%%%%%%%%%%%%

%%%%%%% Protect against orhpans and widows %%%%%%%%%
\widowpenalty=300
\clubpenalty=300
%%%%%%%%%%%%%%%%%%%%%%%%%%%%%%%



%%%%%% Depracted  %%%%%%%%
%\usepackage{en-bib}  % anyone who knows this?
%\usepackage{en-bib}  % anyone who knows this?
%%%%%%%%%%%%%%%%%%%%


%%%%%%%%% Make links work %%%%%%%%%%
\usepackage[pdfborder={0 0 0 0}, backref=none]{hyperref}
%\usepackage[a4paper, bookmarks=true, bookmarksopen=true, bookmarksnumbered=true, pdfborder={0 0 0 0}, colorlinks=true, breaklinks=true, backref=section]{hyperref}
\hypersetup{
pdfborder=0 0 0
}
%%%%%%%%%%%%%%%%%%%%%%%%%%%%%%


%%% something? useful? hopefully!
%\usepackage[]{graphicx} %dvips
\usepackage{emp}
%\ifx\pdftexversion\undefined
%\usepackage[dvips]{graphicx}
%\else
%\usepackage[pdftex]{graphicx}
%\DeclareGraphicsRule{*}{mps}{*}{}
%\fi
\usepackage[]{subfig}% Need to make several pictures in one float
\usepackage{wrapfig}% Enables us to wrap text around a figure
%%%%%%%%%%%%%%%%%%%%%%%%%%%%%


%%%%%% Prettier chapters %%%%%
%\usepackage[Lenny]{fncychap}%
%\usepackage[Sonny]{fncychap}%
%\usepackage[Glenn]{fncychap}%
%\usepackage[Bjarne]{fncychap}%
%\usepackage[Bjornstrup]{fncychap}%
%\usepackage[Conny]{fncychap}%
%\usepackage[Rejne]{fncychap}%
\usepackage{xparse}%
%%%%%%%%%%%%%%%%%%%%



%\setcitestyle{numbers}

%%%% Bibliography %%%%%%
\bibliographystyle{plainnat}
%%%%%%%%%%%%%%%%%


%%%%%% Something something might be important  %%%%%%%%%
\setcounter{secnumdepth}{3}
\setcounter{tocdepth}{2}
\linespread{1}
%%%%%%%%%%%%%%%%%%%%

%%%%%%%%% Depracted %%%%%%%%%%%
%\setlength{\marginparwidth}{10pt}
%\setlength{\textwidth}{400pt}
%\setlength{\textheight}{620pt}
%\setlength{\voffset}{0pt}
%\setlength{\hoffset}{0pt}
%\setlength{\topmargin}{0pt}
%\setlength{\headsep}{10pt}
%\setlength{\oddsidemargin}{50pt}
%\setlength{\evensidemargin}{10pt}
%%%%%%%%%%%%%%%%%%%%



%%%%%%%%%%%%  COMMANDS   %%%%%%%%%%%%%%%%%%
%%%%%%%%%%%%%%%%%%%%%%%%%%%%%%%%%%%%%%%%%%%
%%%%%%%%%%%%%%%%%%%%%%%%%%%%%%%%%%%%%%%%%%%
%%%%%%%%%%%%%%%%%%%%%%%%%%%%%%%%%%%%%%%%%%%

%%%%%%%%%% Get commands defined Elsewhere %%%%%%%%%
%% c = first letter capital
% cap = all capital
% i = italic
% b = bold
% ci = first letter cap and all italic.
 \newcommand{\theWord}{some}
\newcommand{\caseControl}[4]{%
     \ifthenelse{\equal{#3}{c}}% First letter capital
    {\renewcommand{\theWord}{\MakeUppercase{#1}#2}}% if #1 true
    {}% if #1 false
     \ifthenelse{\equal{#3}{ci}}% first letter cap and italic
    {\renewcommand{\theWord}{\textit{\MakeUppercase{#1}#2}}}% if #1 true
    {}% if #1 false
      \ifthenelse{\equal{#3}{ic}}%% first letter cap and italic
    {\renewcommand{\theWord}{\textit{\MakeUppercase{#1}#2}}}% if #1 true
    {}% if #1 false
      \ifthenelse{\equal{#3}{cb}}%% first letter cap and italic
    {\renewcommand{\theWord}{\textbf{\MakeUppercase{#1}#2}}}% if #1 true
    {}% if #1 false
        \ifthenelse{\equal{#3}{bc}}%% first letter cap and italic
    {\renewcommand{\theWord}{\textbf{\MakeUppercase{#1}#2}}}% if #1 true
    {}% if #1 false
   \ifthenelse{\equal{#3}{cbi}}%% first letter cap and italic
    {\renewcommand{\theWord}{\textbf{\MakeUppercase{#1}#2}}}% if #1 true
    {}% if #1 false
        \ifthenelse{\equal{#3}{bci}}%% first letter cap and italic
    {\renewcommand{\theWord}{\textbf{\MakeUppercase{#1}#2}}}% if #1 true
    {}% if #1 false
   \ifthenelse{\equal{#3}{icb}}%% first letter cap and italic
    {\renewcommand{\theWord}{\textbf{\MakeUppercase{#1}#2}}}% if #1 true
    {}% if #1 false
        \ifthenelse{\equal{#3}{ibc}}%% first letter cap and italic
    {\renewcommand{\theWord}{\textbf{\MakeUppercase{#1}#2}}}% if #1 true
    {}% if #1 false
   \ifthenelse{\equal{#3}{cib}}%% first letter cap and italic
    {\renewcommand{\theWord}{\textbf{\MakeUppercase{#1}#2}}}% if #1 true
    {}% if #1 false
        \ifthenelse{\equal{#3}{bic}}%% first letter cap and italic
    {\renewcommand{\theWord}{\textbf{\MakeUppercase{#1}#2}}}% if #1 true
    {}% if #1 false
   \ifthenelse{\equal{#3}{u}}%% first letter cap and italic
    {\renewcommand{\theWord}{\textbf{\MakeUppercase{#1}#2}}}% if #1 true
    {}% if #1 false
        \ifthenelse{\equal{#3}{ub}}%% first letter cap and italic
    {\renewcommand{\theWord}{\textbf{\MakeUppercase{#1}#2}}}% if #1 true
    {}% if #1 false
   \ifthenelse{\equal{#3}{bu}}%% first letter cap and italic
    {\renewcommand{\theWord}{\textbf{\MakeUppercase{#1}#2}}}% if #1 true
    {}% if #1 false
        \ifthenelse{\equal{#3}{biu}}%% first letter cap and italic
    {\renewcommand{\theWord}{\textbf{\MakeUppercase{#1}#2}}}% if #1 true
    {}% if #1 false
      \ifthenelse{\equal{#3}{i}}%% italic
    {\renewcommand{\theWord}{\textit{#1#2}}}% if #1 true
    {}% if #1 false
      \ifthenelse{\equal{#3}{b}}%% bold
    {\renewcommand{\theWord}{\textbf{#1#2}}}% if #1 true
    {}% if #1 false
     \ifthenelse{\equal{#3}{cap}}% all cap
    {\renewcommand{\theWord}{\MakeUppercase{#1#2}}}% if #1 true
    {}% if #1 false
       \ifthenelse{\isempty{#3}}% %if nothing is stated 
    {\renewcommand{\theWord}{#1#2}}% if #1 true
    {}% if #1 false 
    \ifthenelse{\equal{\theWord}{some}}% % Double check actually
    {\renewcommand{\theWord}{#1#2 }}% if #1 true
    {}% if #1 false 
    %%%%%% Standard definitions of words %%%%%%
        \ifthenelse{\equal{#4}{i}}% 
    {\textit{\theWord}}% if #1 true  % Print it italic
    {}% if #1 false 
	    \ifthenelse{\equal{#4}{b}}% Print it in bold
    {\textbf{\theWord}}% if #1 true
    {}% if #1 false 
        \ifthenelse{\equal{#4}{u}}% print it underlinde (not working yet)
    {{\theWord}}% if #1 true
    {}% if #1 false 
       \ifthenelse{\isempty{#4}}%  If there is nothing stated here just print the shit. 
    {\theWord}% if #1 true
    {}% if #1 false    
    \renewcommand{\theWord}{some}%
  }
\newcommand{\myMonth}{Some}
\newcommand{\myDate}[3]{%
\ifthenelse{\equal{#2}{1}}%
{\renewcommand{\myMonth}{January}}{}%
\ifthenelse{\equal{#2}{2}}%
{\renewcommand{\myMonth}{February}}{}%
\ifthenelse{\equal{#2}{3}}%
{\renewcommand{\myMonth}{Marts}}{}%
\ifthenelse{\equal{#2}{4}}%
{\renewcommand{\myMonth}{April}}{}%
\ifthenelse{\equal{#2}{5}}%
{\renewcommand{\myMonth}{May}}{}%
\ifthenelse{\equal{#2}{6}}%
{\renewcommand{\myMonth}{June}}{}%
\ifthenelse{\equal{#2}{7}}%
{\renewcommand{\myMonth}{July}}{}%
\ifthenelse{\equal{#2}{8}}%
{\renewcommand{\myMonth}{August}}{}%
\ifthenelse{\equal{#2}{9}}%
{\renewcommand{\myMonth}{September}}{}%
\ifthenelse{\equal{#2}{10}}%    
{\renewcommand{\myMonth}{October}}{}%
\ifthenelse{\equal{#2}{11}}%
{\renewcommand{\myMonth}{November}}{}%
\ifthenelse{\equal{#2}{12}}%
{\renewcommand{\myMonth}{December}}{}%
\ifthenelse{\isempty{#1}}%
{\myMonth{} #3}%
{\myMonth{} #1, #3}}%
%% This is where we define words. 
%% The last parameter should be blank as standard, but you can add i,b,u. 

%% WORD LIST
%% This is where we define words. 
%% The last parameter should be blank as standard, but you can add i,b,u. 


\newcommand{\admpers}[1][]{\caseControl{a}{dministrative personnel}{#1}{}}%

\newcommand{\block}[1][]{\caseControl{M}{oodle block}{#1}{}}%
\newcommand{\groupname}{Project Group Team}
\newcommand{\systemname}[1][]{\caseControl{M}{yMoodle}{#1}{}}%
\newcommand{\admlib}[1][]{\caseControl{a}{dmin tool library}{#1}{}}%
\newcommand{\subsystem}[1][]{\caseControl{s}{ub-system}{#1}{}}%
\newcommand{\subgroup}[1][]{\caseControl{s}{ub-group}{#1}{}}%
\newcommand{\sn}[1][]{\systemname[]}
\newcommand{\systemmode}[1][]{\caseControl{``F}{ollow mode''}{#1}{}}%
\newcommand{\sm}[1][]{\systemmode[#1]}%
\newcommand{\viewroom}[1][]{\caseControl{v}{iew}{#1}{}}%

%car names
\newcommand{\leader}[1][]{\caseControl{S}{pinky}{#1}{}}%
\newcommand{\follower}[1][]{\caseControl{T}{he Dutch Rammer}{#1}{}}%


\newcommand{\rdesc}[1][]{\caseControl{r}{ecursive-descent}{#1}{}}%
\newcommand{\mas}[1][]{\caseControl{m}{ulti agent system}{#1}{}}%
\newcommand{\idt}[1][]{\caseControl{i}{dentification table}{#1}{}}%
\newcommand{\agd}[1][]{\caseControl{A}{rongadongk}{#1}{}}%
\newcommand{\abst}[1][]{\caseControl{a}{bstract syntax tree}{#1}{}}%

\newcommand{\ooad}[1][]{OOAD}%
\newcommand{\ainterface}[1][]{\caseControl{A}{dmin Interface}{#1}{}}%
\newcommand{\cinterface}[1][]{\caseControl{C}{lient Interface}{#1}{}}%
\newcommand{\sinterface}[1][]{\caseControl{S}{taff Interface}{#1}{}}%
\newcommand{\helpdesk}[1][]{\caseControl{H}{opla Helpdesk}{#1}{}}%
\newcommand{\hdesk}[1][]{\caseControl{H}{opla Helpdesk}{#1}{}}
%\newcommand{\wmon}[1][]{\caseControl{w}{orkload monitor}{#1}{}}
\newcommand{\staff}[1][]{\caseControl{s}{taff}{#1}{}}
\newcommand{\client}[1][]{\caseControl{c}{lient}{#1}{}{}}%
\newcommand{\astaff}[1][]{\caseControl{s}{taff}{#1}{}}
\newcommand{\aclient}[1][]{\caseControl{c}{lient}{#1}{}{}}%
\newcommand{\sadmin}[1][]{\caseControl{a}{dmin}{#1}{}}
\newcommand{\admin}[1][]{\caseControl{a}{dmin}{#1}{}}
\newcommand{\ucsproblem}[1][]{\caseControl{s}{ubmit problem}{#1}{}}
\newcommand{\ucsolproblem}[1][]{\caseControl{s}{olve problem}{#1}{}}
\newcommand{\closed}[1][]{\caseControl{c}{losed}{#1}{}}
\newcommand{\open}[1][]{\caseControl{u}{nsolved}{#1}{}}
\newcommand{\solved}[1][]{\caseControl{s}{olved}{#1}{}}
\newcommand{\unsolved}[1][]{\caseControl{a}{ctive}{#1}{}}
\newcommand{\pinactive}[1][]{\caseControl{c}{losed}{#1}{}}
\newcommand{\pactive}[1][]{\caseControl{a}{ctive}{#1}{}}
\newcommand{\gstat}[1][]{\caseControl{s}{tatistics}{#1}{}}
\newcommand{\bloadwork}[1][]{\caseControl{b}{alance loadwork}{#1}{}}
\newcommand{\worklist}[1][]{\caseControl{w}{orklist}{#1}{}}
\newcommand{\todolist}[1][]{\caseControl{w}{orklist}{#1}{}}
\newcommand{\problem}[1][]{\caseControl{p}{roblem}{#1}{}}
\newcommand{\tucadmin}[1][]{\caseControl{a}{dministrate}{#1}{}}
\newcommand{\sql}[1][]{SQL} % just in case...
\newcommand{\aspnet}[1][]{ASP.NET} % just in case...
\newcommand{\posgresql}[1][]{PostgreSQL}
\newcommand{\iis}[1][]{IIS}
\newcommand{\wholeiis}[1][]{Microsoft Internet Information Services}
\newcommand{\spinterface}[1]{\caseControl{s}{olve admin interface}{#1}{}}
\newcommand{\viewmodel}[1]{\caseControl{v}{iew model}{#1}{}}


\newcommand{\Hdesk}[1][]{\hdesk[c]}
\newcommand{\Wmon}[1][]{\wmon[c]}
\newcommand{\Staff}[1][]{\staff[c]}
\newcommand{\Client}[1][]{\client[c]}
\newcommand{\Astaff}[1][]{\astaff[c]}
\newcommand{\Aclient}[1][]{\aclient[c]}
\newcommand{\Helpdesk}[1][]{\helpdesk[c]}
\newcommand{\Sadmin}[1][]{\sadmin[c]}
\newcommand{\Admin}[1][]{\admin[c]}

\newcommand{\john}[1][]{\caseControl{j}{ohn}{#1}{}}
\newcommand{\michael}[1][]{\caseControl{M}{ikael}{#1}{}}
\newcommand{\rubik}[1][]{\caseControl{R}{ubik's Cube}{#1}{}}
\newcommand{\facet}[1][]{\caseControl{f}{acelet}{#1}{}}
\newcommand{\facelet}[1][]{\caseControl{f}{acelet}{#1}{}}
\newcommand{\cube}[1][]{\caseControl{R}{ubik's Cube}{#1}{}}
\newcommand{\cuber}[1][]{\caseControl{c}{uber}{#1}{}}
\newcommand{\face}[1][]{\caseControl{f}{ace}{#1}{}}
\newcommand{\cpiece}[1][]{\caseControl{c}{ubie}{#1}{}}% Not that the one below are the same as this one
\newcommand{\cubie}[1][]{\caseControl{c}{ubie}{#1}{}}%
\newcommand{\cubicle}[1][]{\caseControl{c}{ubicle}{#1}{}}
\newcommand{\twist}[1][]{\caseControl{t}{wist}{#1}{}}
%\newcommand{\turn}[1][]{\caseControl{t}{urn}{#1}{}}
%\newcommand{\rotate}[1][]{\caseControl{r}{otate}{#1}{}}
\newcommand{\erno}[1][]{\caseControl{E}{rn\"{o} Rubik}{#1}{}}
\newcommand{\mpuzzle}[1][]{\caseControl{M}{agic Puzzle}{#1}{}}
\newcommand{\msquare}[1][]{\caseControl{M}{agic Square}{#1}{}}
\newcommand{\mcube}[1][]{\caseControl{M}{agic Cube}{#1}{}}

 

%\input{extra}
%%%%%%%%%%%%%%%%%%%%

%%%%  Makes the titles look nicer. i guess. Rasmus out. %%%%%%% Don't remove
%\usepackage{titlesec} \newcommand{\bigrule}{\titlerule[0.5mm]} \titleformat{\chapter}[display] {\bfseries\Huge} {  \vskip-2em 
 %\titlerule 
% \filright  \huge\chaptertitlename\ \vspace{5mm}  \huge\thechapter} {0mm} {\filright} [\vspace{3mm} \bigrule \vspace{-10mm}] %
%%%%%%%%%%%%%%%%%%%%%%%%%%%%

%%%%% Depracted %%%%%%%%%
\newcommand{\picturepath}[1]{input/pics/}


%%%%% Used to determine the highlight of the first word in the terminology %%%%%
\newcommand{\myTermHigh}[1]{\textbf{#1}: }
%%%%%%%%%%%%%%%%%%%%%%

%%%%%%%%% tops 'n' tails %%%%%%%%%%%%
\newcommand{\myTop}[1]{\vspace{-8mm}  \vspace{0mm} \textit{#1} \vspace{2.6mm} \hrule  \vspace{4.2mm} }

%\newcommand{\myTop}[1]{\vspace{-8mm}  \vspace{0mm} \textit{#1} \vspace{3.4mm} \hrule \vspace{3.4mm} }
%\newcommand{\myTop}[1]{ \vspace{3.4mm} \textit{#1} \  \\  \hrule \  \\}%
\newcommand{\myTail}[1]{ \vspace{3.4mm} \hrule \vspace{3.4mm} \textit{#1} }%

\newcommand{\emptyTop}[0]{\vspace{-6mm}}
%%%%%%%%%%%%%%%%%%%%%%%%%%%%

%%%%%%%%% Use this for caption text %%%%%%%%%%
\newcommand{\myCaption}[1]{\textit{\footnotesize #1.}}
\newcommand{\morscaption}[1]{\caption{\myCaption{#1}}}
%\usepackage[bf]{caption}
%%%%%%%%%%%%%%%%%%%%%%%%%%%%

%%%%%%%% Bruges til skillekolonner og r\ae{}kker . Definerer tykkelsen. %%%%%%
\newcommand{\vrules}{{\vrule width 0.6pt}}
\newcommand{\hrules}{{\hrule height 1.2pt}}
%%%%%%%%%%%%%%%%%%%%%%%%%%%%%%%%%%%%%%%%%%%

%%%%%%%Commands for getting e.g. ``st'' lifted in 1st   %%%%%%%%%%%
\newcommand{\superscript}[1]{\ensuremath{^{\textrm{#1}}}}
\newcommand{\subscript}[1]{\ensuremath{_{\textrm{#1}}}}
\newcommand{\ths}[0]{\superscript{th}}
\newcommand{\st}[0]{\superscript{st}}
\newcommand{\nd}[0]{\superscript{nd}}
\newcommand{\rd}[0]{\superscript{rd}}
%%%%%%%%%%%%%%%%%%%%%%%%%%%%%%%%%%%%%%%%%%%

%%%%%%%%% FIXME macro %%%%%%%%%%%%
%\newcommand{\todo}[1]{\fxnote{#1}}
%\newcommand{\todoV}[1]{\fxfatal{#1}}
%\newcommand{\todov}[1]{\fxfatal{#1}}

%%%%%%%%%%%%%%%%%%%%%%%%%%%%%%%%%%

%%%%%%%%% REFERENCES %%%%%%%%%%%%%
\newcommand{\agdref}[2]{#1~\ref{#2}}

%%%%%%%%%%%%%%%%%%%%%%%%%%%%%%%%%%

%%%%%%%%%% Moves %%%%%%%%%%
\newcommand{\m}[1]{\textbf{#1}}
\newcommand{\vr}[1]{$#1$}
\newcommand{\pcparagraph}[1]{\vspace{-1.5mm}\paragraph{#1}}
%%%%%%%%%%%%

%%%%%%%%%%%%%%%%%%%%%%%%%%%


\newcommand{\OK}[0]{\textsc{Ok}}
\newcommand{\Ok}[0]{\textsc{Ok}}
\newcommand{\Undef}[0]{\textsc{Undefined}}
\newcommand{\Wrong}[0]{\textsc{Wrong}}
\newcommand{\Nil}[0]{\textsc{Nil}}
\newcommand{\sensorDist}[0]{15 cm}

%%%%%%%%%%%%%%%%%%%%%%%%%%%%%%%%
% JOURNALS %%
\newcommand{\journalwidth}[0]{\textwidth}


%%%%%%%%%%%%%%%%%%%%%%%%%%%

%%%%%%%%%% Class, component ... %%%%%%%%%%%%%%%%%%

\newcommand{\me}[1]{\method{#1}}							%%Don't change
\newcommand{\class}[1]{\textbf{#1}}						%%Do change
\newcommand{\cl}[1]{\class{#1}}								%%Don't change
\newcommand{\aclass}[1]{\textit{\class{#1}}}	%%Do change
\newcommand{\acl}[1]{\aclass{#1}}							%%Don't change
\newcommand{\component}[1]{\textbt{#1}}				%%Do change
\newcommand{\comp}[1]{\component{#1}}					%%Don't change
\newcommand{\vari}[1]{\mbox{\textbf{#1}}}						%%Do change
\newcommand{\method}[1]{\mbox{\texttt{#1}}}						%%Do change
\newcommand{\function}[1]{\method{#1}}				%%Do change
\newcommand{\fu}[1]{\function{#1}}						%%Don't change
\newcommand{\figref}[1]{Figure~\ref{#1}}			%%Do change
\newcommand{\coderef}[1]{Code Snippet~\ref{#1}}	%%Do change
\newcommand{\chapref}[1]{Chapter~\ref{#1}}			%%Do change
\newcommand{\secref}[1]{Section~\ref{#1}}			%%Do change
\newcommand{\appref}[1]{Appendix~\ref{#1}}			%%Do change


\newcommand{\type}[1]{\textbf{#1}} % added by Stubman 110420
\newcommand{\object}[1]{\texttt{#1}} % added by Stubman 110420

%%%%%%%%%%%%%%%%%%%%%%%%%%%%%%%%%%%%%%%%%%%%%%%%%%

%%%%%%%%%%%%%%% Qoutations %%%%%%%%%%%%
\newcolumntype{R}{>{\raggedleft\arraybackslash}X}%
\newcommand{\myQuote}[1]{
\begin{flushleft}
\Huge{\textbf{``}}
\end{flushleft}
\vspace{-30pt}
\begin{quote}
#1
\end{quote}
\vspace{-30pt}
\begin{flushright}
\Huge{\textbf{''}}
\end{flushright}
}


%%%%%%%%% Defining Theorem %%%%%%%%%%%
\theoremstyle{definition} \newtheorem{theorem}{Theorem}
%%%%%%%%%%%%%%%%%%%%%%%%%%%%%%%%%%%%%%%%%%%

%%%%%%%%% Multi Row %%%%%%%%%%%%%%%%%%%
\usepackage{multirow}
%%% This is needed to use the multicolumn command
%%%%%%%%%%%%%%%%%%%%%%%%%%%%%%%%%%%%%%%%%%%
\hyphenation{help-desk}

%%%%%%%%%%%%%%%%%%%%%%%%%%%%%%%%%%%%%%%%%%%%%%%%
%%% Environments for operational semantics
%%%%%%%%%%%%%%%%%%%%%%%%%%%%%%%%%%%%%%%%%%%%%%%%
\newcommand{\envv}{\textit{env}_{v}}
\newcommand{\envf}{\textit{env}_{f}}
\newcommand{\sto}{\textit{sto}}
\newcommand{\ret}{\textit{ret}}
\newcommand{\Void}{\textsc{Void}}
\newcommand{\Int}{\textit{Int}}
\newcommand{\Float}{\textit{Float}}
\newcommand{\Bool}{\textit{Bool}}
\newcommand{\String}{\textit{String}}
\newcommand{\Types}{\textbf{Types}}
\newcommand{\semspace}{\vspace{-6mm}}

%%%%%%%%% List Environments %%%%%%%%%%%
\usepackage{listings}


\usepackage{color}

\definecolor{light-gray}{gray}{0.80}
\definecolor{gray95}{gray}{.95}
\definecolor{gray92}{gray}{.92}
\definecolor{gray75}{gray}{.75}
\definecolor{gray45}{gray}{.45}

\lstdefinestyle{arongadongkCode}
{ 
	%numbers=left,
	numbersep=5pt, 
	stepnumber=1,
	captionpos=b,  %bottom
	keywordstyle=\color[rgb]{0,0,1},
	commentstyle=\color[rgb]{0.133,0.545,0.133},
	stringstyle=\color[rgb]{0.627,0.126,0.941},
	%backgroundcolor=\color{gray95},
	%frame=lrtb,
	framerule=0.5pt,
	linewidth=1.00\textwidth,
	tabsize=4,
	numberbychapter=true,
	basicstyle=\ttfamily\footnotesize,
	breaklines=true,
	showstringspaces=false,
	morecomment=[l]{//},
	morecomment=[s]{/*}{*/},
	emph=[1]{int,string,float,bool},%%%%%%%%%%% Types
	emph=[2]{global,const,if,while,else,return}, %%%%%%%%%%% Keywords
	emphstyle=[1]{\color[rgb]{0,0.7,0.7}},
	emphstyle=[2]{\color[rgb]{0.7,0,1}},
	float=htb,
	breakindent=20pt
}

\lstdefinestyle{sourceCode}
{ 
	numbers=left,
	numbersep=5pt, 
	stepnumber=1,
	captionpos=b,  %bottom
	keywordstyle=\color[rgb]{0,0,1},
	commentstyle=\color[rgb]{0.133,0.545,0.133},
	stringstyle=\color[rgb]{0.627,0.126,0.941},
	%backgroundcolor=\color{gray95},
	%frame=lrtb,
	framerule=0.5pt,
	linewidth=1.00\textwidth,
	tabsize=4,
	numberbychapter=true,
	basicstyle=\ttfamily\footnotesize,
	language=C,
	breaklines=true,
	showstringspaces=false,
	emph=[1]{endregion,region,get,set,enum},%%%%%%%%%%% Add new keywords here
	%emph=[2]{Tag,Problem,Person,List,NotSupportedException,TestMethod,ProblemSearch,Assert,
	%EntityCollection,Department,IEnumerable,TimeSpan,DateTime},%%Classes
	emphstyle=[1]{\color[rgb]{0,0,1}},
	emphstyle=[1]{\color[rgb]{0,0,1}},
	emphstyle=[2]{\color[rgb]{0.1,0.5,0.5}},
	float=htb,
	breakindent=20pt
}
\lstdefinestyle{phpCode}
{ 
	numbers=left,
	numbersep=5pt, 
	stepnumber=1,
	captionpos=b,  %bottom
	keywordstyle=\color[rgb]{0,0,1},
	commentstyle=\color[rgb]{0.133,0.545,0.133},
	stringstyle=\color[rgb]{0.627,0.126,0.941},
	%backgroundcolor=\color{gray95},
	%frame=lrtb,
	framerule=0.5pt,
	linewidth=1.00\textwidth,
	tabsize=4,
	numberbychapter=true,
	basicstyle=\ttfamily\footnotesize,
	language=PHP,
	breaklines=true,
	showstringspaces=false,
	emph=[1]{endregion,region,get,set,enum},%%%%%%%%%%% Add new keywords here
	%emph=[2]{Tag,Problem,Person,List,NotSupportedException,TestMethod,ProblemSearch,Assert,
	%EntityCollection,Department,IEnumerable,TimeSpan,DateTime},%%Classes
	emphstyle=[1]{\color[rgb]{0,0,1}},
	emphstyle=[1]{\color[rgb]{0,0,1}},
	emphstyle=[2]{\color[rgb]{0.1,0.5,0.5}},
	float=htb,
	breakindent=20pt
}

\lstdefinestyle{oilCode}
{ 
	%numbers=left,
	numbersep=5pt, 
	stepnumber=1,
	captionpos=b,  %bottom
	keywordstyle=\color[rgb]{0,0,1},
	commentstyle=\color[rgb]{0.133,0.545,0.133},
	stringstyle=\color[rgb]{0.627,0.126,0.941},
	%backgroundcolor=\color{gray95},
	%frame=lrtb,
	framerule=0.5pt,
	linewidth=1.00\textwidth,
	tabsize=4,
	numberbychapter=true,
	basicstyle=\ttfamily\footnotesize,
	language=C,
	breaklines=true,
	showstringspaces=false,
	emph=[1]{Task},%%%%%%%%%%% Add new keywords here
	%emph=[2]{Tag,Problem,Person,List,NotSupportedException,TestMethod,ProblemSearch,Assert,
	%EntityCollection,Department,IEnumerable,TimeSpan,DateTime},%%Classes
	emphstyle=[1]{\color[rgb]{0,0,1}},
	emphstyle=[1]{\color[rgb]{0,0,1}},
	emphstyle=[2]{\color[rgb]{0.1,0.5,0.5}},
	breakindent=20pt
}

%\renewcommand{\follower}{Ju{}dge John{}son}%Asure uncorrect noname :P Bondo you DAAUGH
\renewcommand{\lstlistingname}{Code Snippet}%%Changing the caption to read ``Code snippet''
\renewcommand{\lstlistlistingname}{List of Code Snippets}
\lstset{escapeinside={(*}{*)}}%%Defines the escape and unescape chars

%%%%%%%%%% To use with copy-paste:
\begin{comment}

\begin{lstlisting}[style=sourceCode, caption=\myCaption{<some caption>}, label=<some label>]
<the code>
	<more code, now with indent>
\end{lstlisting}

\end{comment}
%%%%%%%%%% To input file:
\begin{comment}

\lstinputlisting[style=sourceCode, caption=\myCaption{<some caption>}, label=<some label>]{<file name>}

\end{comment}


\newcommand{\moodlefile}[1]{#1}
%\newcommand{\sharedReport}[0]{../../sw6-mymoodle-group/shared_report/}
%\input{\sharedReport preamble}