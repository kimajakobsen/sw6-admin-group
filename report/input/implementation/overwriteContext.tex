
\subsection{Overwrite Context}
In \secref{sub:contextsystem}  the context system of Moodle is described. In this section the creation of a custom context is described. 

To be able to define capabilities for the project groups and have per project group blocks we need our own context. We will create our own context level and class.
We call the context for \cl{context\_projectgroup}. 

Moodle does not support extension of contexts through one of the more than 30 different plugin types available \cite{moodleplugins}. 
There two parts of the problem, the first is to create a new context and the second is to load it properly. 
We create a new context by making a new class which is very similar to context\_course class and by defining the context level as an constant. 
The class header and the constant definition can be seen in \coderef{codeprojectgroupcontext}. 
The constant is set to 55 and is chosen because that context level is unused and it is close to the course context level. 
We regard the projectgroup to be at same equivalent level of the courses. 

\begin{lstlisting}[style=phpCode, caption=\myCaption{The context\_projectgroup class header and constant definition}, label=codeprojectgroupcontext]
define ('CONTEXT_PROJECTGROUP',55);
class context_projectgroup extends context {
\end{lstlisting}

When a context is loaded in a Moodle page it is instantiated by \fu{get\_context\_instance}, which takes a context level and an instance id. 
The instance id can be a course id or similar dependent on the context. 
This function calls a static method in the \cl{context\_helper} class which uses a private array to translate the context level into a class name.
The overall system definition in \chapref{chap:systemDef} retain us from changing the core code of Moodle. 
If this constraint were not enforced the array could simply be extended directly in the code.  
Since the array used is private we can not extend the context system by overriding the \cl{context\_helper} class. 
The newly created context is only used in pages created in this project and we can therefore create our own version of \fu{get\_context\_instance}. 
The new function can be seen in \coderef{codeprojectgroupcontextinstance}.
\begin{lstlisting}[style=phpCode, caption=\myCaption{The function to get projectgroup context}, label=codeprojectgroupcontextinstance]
function get_projectgroup_context_instance( $instance = 0, $strictness = IGNORE_MISSING) 
{ 
    return context_projectgroup::instance($instance, $strictness);
}
\end{lstlisting}
The \vari{instance} variable denotes a project group id. 

With the new context and the function to instantiate it we can now make per project group permissions and add blocks to specific project group pages. 