\chapter{Introduction}
\label{chap:introProjectgroup}
%Imagine an LMS that incorporates course management and the Aalborg PBL model along side each other.
%This is the main idea behind this project.
The foundation for this project has been set up in Part I.
We now proceed to the part were the actual development is presented.
The \subsystem{} that we are developing is called \administrationgroup{} and is briefly described in \secref{sub:decomposingSys}.


%As of this part only the authors noted in the title page have been working actively on this report, with help from our supervisor.
%Hence ``we'' only refers to the four authors of this report, and not the 14 collaborators of the entire system.

From this part on the report has four authors; ``we'' refers to those four authors.

During the development of this project we use an agile development method, which is explained further in \chapref{sec:developmentpractice}. 
The report structure and content does not reflect the use of an agile method and is written as it was developed using a traditional development method. 
We structure the report in this manner to make it more readable and comprehensible. 

It should be noted that the entire system, \system{}, is supposed to be a complete system, not a set of independent systems.
This means that we are referring to the other \subsystem{}s whenever there is a dependence on them or we are making services that the other \subsystem{}s depend on.

We refer to the students working on the other sub-projects as our ``peer-groups''.

%This project is concerned with integrating the Aalborg PBL model, defined in \secref{sec:problemDef}, into \moodle.
%An important concept of this model is the ``Team'' (see \secref{sub:aaupbl}).
%Recall \secref{sub:aaupbl} where the concept ``Team'' was presented, we will now use it to define the term ``project group''.
%Our sub-group has the responsibility to develop a \subsystem{} that allows a team to work together through \moodle{} and some way to manage who is part of which teams with what roles.
%We use the term ``Project Group'' to describe a project with a belonging team.
%The relation between a team and a project will be define in \secref{sec:projectgroup}.
%We hence call this tool ``administration tool''.

As we are working with \administrationgroup{} we are responsible for integrating the concept project group into \moodle{}. 
Recall from \secref{sub:aaupbl} that we define a project group to be a team of students working on a project.


%For a team to operate in an interactive environment it is necessary to have some entity that defines the members of the team and the actions that they can perform to solve their problem.
%We call this entity a ``Project Group''.

When referring to a ``block'' we mean a specific plugin type in Moodle and not as a general concept.

The members of a project group will have access to shared tools through which they can conduct their project work.
These tools include planning the progress of the project, communicating internally while the project is being conducted, and communicating with supervisors.
These tools are created by our peer-groups (as mentioned in \secref{sub:decomposingSys}).
Our responsibility is to make all these tools available to the members of the project groups.
To make \system{} usable at any educational institution we provide a tool to manage project groups.



