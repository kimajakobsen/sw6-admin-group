\section{Requirements}
\label{sec:requirements}
During this project we have had several meetings with people that we consider to be our end users.
Some of the meetings were held before we coded the system and some of them during coding.
In this section we present our the final requirements.
The continues change in our requirements is evaluated in \secref{sec:intergroup} in \partref{part:evaluation}.
The interviews that have been conducted to gather these requirements are found in \appref{app:interview}.

Since we use scrum as our main development practice we use a product backlog to capture our requirements~\cite[p.~114]{Larman04}.
Different sources suggest different notations for what is in a backlog~\cite[p.~17]{scrumchecklist}\cite[pp.~123-124]{Larman04}.
Our backlog items can be are all use cases, features, or fixes.
Since this project is to be proceeded by a new group of students next year, we allow our product backlog to be larger than we are able to complete in this project.
We will now focus on the items that are part of this semester project.
These are said to be part of our release backlog.
The backlog items that make up our product backlog for our \subsystem{} are defined in \figref{fig:productbacklog}.
%The process' behind creation of the most central backlog items are also presented here.
We will now describe how we derived some of our backlog items.

\begin{figure}%
\begin{tabular}{|p{0.2\textwidth}|p{0.8\textwidth}|}
	\hline
	
	\textbf{Name} & \textbf{Description} \\
	\hline
	Manage Project Groups & A subset of our target group should be able to manage the project groups, this includes creating, editing, and deleting project groups.  \\
	\hline
	Virtual Meeting Place & Create a virtual space that are compliant with the principals of the Aalborg PBL model and create a feeling of nearness.  \\
	\hline
	\textit{Recursive Project Groups} & \textit{Allow project groups to contain groups of persons. This also include the administrative tools to manage the groups and sub-groups.} \\
	\hline
	Introduce Roles & Add the possibility to assign roles to the members of a project group, roles such as supervisor and student.   \\
	\hline
	\textit{Manage Roles} & \textit{Create Functionality to define and manages roles. An example of a new role  could be secondary supervisor.} \\
	\hline
	Find Project Groups & Produce functionality to find a given project group. This involves creating a list of project groups and making features to allow search or filtering. \\
	\hline
	Navigate to Project Groups & The members of the project group should have a way to navigate to it. \\
	\hline
	Project Group Members & It should be possible to see who the members of the group is, to create the feeling of nearness. \\
	\hline
	\textit{Project Group Synchronization} & \textit{Project groups should reflect the physical project groups at the university. Thus when a student joins an existing group, the student should automatically be added to the virtual group room.}  \\
	\hline 
	\textit{Virtual Meeting Place Template} & \textit{To account for the user groups diversity and different needs, a set of templates, that defines the visual appearance of the virtual meeting place, could be applied when creating a new virtual meeting space.}  \\
	\hline 
	\textit{Archiving Project Group} & \textit{ELSA } \\
	\hline 
\end{tabular}
\morscaption{A list showing features from our product backlog. The italic items are not part of our release backlog}
\label{fig:productbacklog}
\end{figure}

\paragraph{Virtual Meeting Place}
The concept of a virtual meeting place comes from the meeting with ELSA described in \secref{sub:elsaInterview}. 
From the interview with Lene in \appref{sec:lene} we know that she manually types a list of groups and their members into Moodle.
Moodle is not able to interpret this.
In the interview with Jette and Pia (\appref{sec:jettepia}) we learn that a member of the \admpers{} is responsible for the formed groups. 
This implies that there exist records of the composition of the groups.   
In contrast to the these two interviews Morten explains in an interview(\appref{sec:morten}) that on the MPBL education the \admpers{} do not manage the project groups.
Based on this information we decide that we will make a virtual meeting place that reflects the project groups that exist on Aalborg university. 
We decide this knowing that it will cause additional work for the \admpers{} on the MPBL education. 
We will elaborate on the virtual meeting place in \chapref{chap:analdesign}.

\paragraph{Find Project Groups}
We deem that it is necessary to create a list with all the project groups that exist.
This should be available to the part of our target group that manage the project groups.
In our third sprint we held a demo meeting where Lene mentioned that it was difficult to find a project group when there were many project groups in the system(see \appref{sec:lenedemoone}).
We decide to make functionality that will allow users to search in the list of project groups.
The implementation of this concrete problem will be addressed in \secref{sec:manProjGrpImpl}.

\paragraph{Manage Project Groups}
It should be possible to manage the project groups. 
We let us inspire by how courses are currently being created at Aalborg University.
ELSA are responsible for creating new courses at the start of every semester. 
This is done by the \admpers{} from every institute informs ELSA of which courses that should be created. 
ELSA creates the courses based on a general templates.
The \admpers{} and lectures then populates the courses with data. 
Information about this process can be read from \appref{sec:lene}, \secref{sub:elsaInterview},  \appref{sec:jettePia}, and \appref{sec:thomas}.
Similar to the creation of courses it should be possible to create and in other ways to manage project groups. 
The design of project groups will be explained in \chapref{chap:analdesign}. \\

We will not explain all the backlog items because we reckon that the derivation of them are of similar nature.
We will now account for the choices for the release backlog.

\subsection{Release Backlog}
Recall the table seen in \figref{fig:productbacklog}. 
The release backlog is based on our product backlog.
To explain the selection of backlog items we will focus on the items that we do not choose.
These items are marked with italic on the figure.
The four items are described below.

\paragraph{Recursive Project Groups}
In an interview with Lene (see \appref{sec:lenedemoone}), we discussed the possibility of having a recursive group structure.
Lene could not imagine any concrete cases where such a structure would be beneficial. 
An example of a place where there exist groups in groups is our current multi-project.
The 6\ths~semester project of the Software Engineering education anno 2012 is a multi-project, which means that the student are divided into two groups each with its own subject. 
The multi-groups are split into project groups of fitting size.
Despite this we choose to prioritize other features that will bring more value to the project according to our target group and not implement recursive project groups.
We will argue further for this in \secref{} \todo{hvis det er beskrevet i analdesign så indsæt ref. ellers så slet denne  sætning.}

\paragraph{Project Group Synchronization}
In the computer science department of Aalborg university AdmDB is among other things used to manage project groups. 
During the interview with Mikael(see \appref{sec:mikael}) we learned that he is currently developing a new API for the AdmDB system, and it would not be optimal to rely on the old API. \todo{brug fulde navn?}
We also learned that there does not exist a university wide system that contains all project groups. 
We believe that if this feature should be implemented it should be after the new API is finished.
We will elaborate further on this subject in \secref{sub:automanagement}. 

\paragraph{Virtual Meeting Place Template}
As previously mentioned ELSA is responsible for creating course pages in Moodle. 
They do this using a template. 
It is desirable to have the same functionality when creating the virtual meeting place, but this feature will not bring any immediate value to the product. 
We will postpone this feature until there arises a concrete need for such functionality.
We will further discuss this feature in \secref{}.\todo{insert ref when the paragraph is written in the analdesign. If it will not be written delete this line.}\\

The items in the release backlog are the following: Manage Project Groups, Virtual Meeting Place, Introduce Roles, Find Project Groups, Navigate to Project Groups, and Project Group Members. 
Based on our requirements we will now present our \nameref{sec:subSysDef}.
 %use these requirements to make our system statement.

\paragraph{Archiving Project Groups}
\label{sub:analysarchiving}
% Archiving (ELSA, Lene & Jette) (Indbygget archive restore og courses)
From the meeting with ELSA, see \secref{sub:elsaInterview}, we know they archive courses at the end of each semester. 
Lene Winther Even explains that it is to preserve the content for future reference, see \appref{sec:lene}. 
Morten Mathiasen Andersen, see \appref{sec:morten}, explains that teachers use the archived versions of a course to create a new course of the same subject. 
ELSA proposed that project groups can be archived for the same purpose so supervisors and students can look back at their previous work.
%In Moodle it is possible to backup and restore courses, which allows teachers to take a snapshot of their course materials. 
%This feature is not used by ELSA to archive courses.
%Instead they copy the courses and its contents.
%A similar approach can be used to archive project groups for future reference. 

We believe that the archiving of project groups is a very beneficial feature for students, supervisors, and administrative personnel.
However, since we have limited development time we suggest that future student projects could look into this feature.
A further discussion on archiving can be found in \chapref{chap:futurework}.

\begin{comment}
%skriv flere cases af hvordan b-items er opstået 
In the interview described in \appref{sec:lene} we learned that there is a need for having a concept of project groups in \moodle{}.  to enhance communication by sending messages from the \admpers{} to project groups through \moodle{}.
This evolves into two backlog items; integrate the concept of project groups into \moodle{} and allow the administrative personnel to post messages to the project group.
The former is represented as a backlog item which can be seen in \figref{fig:productbacklog}.
The latter is implemented by the \supervisorgroup{}.
\end{comment}

\FloatBarrier
