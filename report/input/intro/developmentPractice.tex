\section{Development Practices} %remember to say how we work with costumers

\subsection{Choosing a Development Method}
%because many people -> agile. our version of scrum
Due to the large number of people that work together in this project, we need a development method that takes this into account.
Furthermore, the development method also needs to take into account that we are working on a platform (namely Moodle), which most of us are unfamiliar with.

We choose to use the development method Scrum\ref{par:scrum} with a few changes.
The reason for why we cannot use Scrum in its original form is that we do not have all the required personnel. 
Scrum requires an on-site costumer and a product owner.
We have neither, so we have to make some modifications to the development method.
Instead we have acquired some customers from which we get requirements and feedback for the system we are developing.
As the \groupname{} we are responsible for creating the administrative part of the system as well as creating the project group page.
Therefore we are in contact with administrative personnel at Aalborg University, which serve as costumers.
The costumers for the functionality of the group page are the three other groups in this multi-project.
How we work together with our costumers will be described later in this section.

\subsection{Utilizing the Development Method}
%utilizing agile methods -> we don't know shit so its good



%sprints

%daily scrum meetings

%weekly casual scrum og scrum meetings

%product backlog
%sprint backlog

%burndown chart

%demo meetings

%sprint planning meetings

%planning poker

%TDD