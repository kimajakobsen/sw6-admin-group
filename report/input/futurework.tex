\chapter{Future Work}
\label{chap:futurework}
\myTop{In this chapter we present the ideas that we believe to be the most valuable additions to \system{}.
Whereas \chapref{chap:evalProduct} focuses on alterations and improvements to the system, this chapter focus on new features and alternative usage of the system.}

%%FYSISKE RUM -> forståelse -> booking

\section{Physical Group Rooms}
\label{sec:booking}
In the current implementation of \system{} there is a concept of group rooms.
This concept is strictly virtual -- there is no understanding of actual physical group rooms.
%If a group has the same physical group room during an entire semester, it would also be possible to identify the group by their group room.

If there were an implementation of physical group rooms in Moodle it would be possible to implement a room booking system.
As mentioned in \secref{sec:virtualMeetingPlace} some groups have to book group rooms daily.
This takes the time of administrators and students.
Such a system could also be expanded to handle the booking of lecture rooms.

There are several constraints that have to be considered before implementing such a system.
First of all there should be priorities based on the person wanting to reserve a room, e.g. a lecturer should take precedence over a student when trying reserve a lecture room.
Additionally there should be a system in place that handles a situation where there are not a sufficient number of group rooms available; some groups should not be able to make a significantly greater number of reservations than some other groups.

We believe such a system has a sufficient complexity and relevance that it could serve as a project for one of next year's $6$\ths~semester software groups.