\section{Inter Group Collaboration}
\label{sec:intergroup}
\todo{comment on change in our requirements}
In this section the chosen development approach and its utilization is discussed.

% Choice of scrum
We chose to use \sos{} as our development method.
We chose this over XP and traditional based methods.
It is difficult to validate that the choice is correct and that \sos{} is the best suited development method compared to the others.
On the other hand; nothing indicates that the other methods should be better suited for our project. 

During our project we have had four sprints, the first one being a non-programming sprint.
In the implementation of \sos{} we chose to have \sos{} meetings nearly twice a week.
%This worked well since they made all groups talk together at least once a week. 
This worked well since they helped bring forth some issues and possibly mitigated many potential risks of misunderstanding between \subgroup{}s.
In the beginning and prior to the employment of a development method we conducted large whole team assemblies. 
This practice worked well in the initial part of the project, but they became ineffective as progress went on. 
A reason for this is that the topics discussed early in the development are important and interesting knowledge to everyone, but later some topics became of no interest to some groups while important to others.
That made the enthusiasm at the big assemblies fall drastically. 


An issue in the implementation of \sos{} is the lack of a product owner, which is described as key concept. 
We compensated for this lack by doing several interviews with possible end users and to have them included in the demo meetings. 
In effect we have had the role of the product owner our selves, which is not an optimal solution. 
%It is not optimal for the reason that it is recognized as bad \scrum{} practice not to have a product owner to prioritize tasks \cite[p.~128]{Larman04}. 
It is not optimal for the reason that it is recognized as bad \scrum{} practice to have many product owners to prioritize and select tasks for a sprint~\cite[p.~128]{Larman04}. 
Furthermore, we have conflicting interests -- we want to implement something exciting and challenging, while the end users may in fact need something we consider to be simple and boring.


Another issue is the lack of a shared room, which is preferred in \sos{}, but it helped that the group rooms are located close to each other and we quickly discovered that letting the door stand open during work hours helped on the inter group communication. 
An example of this is other \subgroup{}s asking about how to get their \block{} displayed in the virtual group room.

In general the development approach worked well and it made us adapt to changes. 
An example of a large change during development was when we realized that it was impossible to use activity modules in project groups without linking them through a course.
This was overcome by changing the already implemented activity modules into \block[]s. 
The changed requirement was embraced and quickly implemented.
To make this change quickly and effective it helped a great deal that our group rooms are located close to each other, since we could communicate with the other project groups as soon as we found the solution.
The solution was also found in collaboration with some of our peer-groups.

\subsection{Scrum Alterations During Development}
Before the first sprint we had a collaborate sprint planning meeting where all tasks for every \subgroup{} was presented and estimated in collaboration.
In successive sprints we decided that every \subgroup{} was responsible for handling these meetings themselves, because there was only a limited gain from sitting in on the selection and estimation of other \subgroup{}s sprint items compared to the gain from \sos{} meetings that we conducted during the sprint.

We never conducted any formal sprint retrospective meetings in collaboration with the other \subgroup{}s.
We compensated for the missing meetings by evaluating the development method along the way.
The \sos{} meetings have also been subjects to use for evaluation of the development method and its practices, this has lead to violation of the time constraint of the \sos{} meetings in some situations.

The demo meetings that we have conducted with our peer-groups have also have some evaluation of the development method along with evaluation of the product.
Ideally we would have conducted our demo meetings with end users immediately after a sprint ended, but in some cases we had to wait because the end users we have been in contact with was not available in the given time period.
This resulted in some feature requests that came in during a sprint and therefore had to be delayed -- we prohibited ourselves from adding work to our sprint backlog during a sprint as \scrum{} suggests.