\chapter*{Preface}
\addcontentsline{toc}{chapter}{\protect\numberline{}Preface}
\label{chap:preface}
This report is the documentation of a bachelor project conducted in the period \myDate{1}{2}{2012} to \myDate{4}{6}{2012} by the software student group sw608f12.
The project is the outcome of the 6\ths{} semester at Aalborg University. 
The overall theme of the multi-project is: ``Application Development''. 
The project is a part of a multi-project consisting of four groups. 
The first part of the report is written in cooperation between the four groups of the multi-project, while this preface and parts two and three are written entirely by our group, who are the participants noted in the title page.

In code snippets throughout the report the notation of three dots (...)  can be seen.
This notation illustrates that one or more lines of code are omitted. 
Code is only omitted in situations where those lines are not important to the description of the code. 

The first few lines of every chapter (written in \textit{italic}) in parts two and three is a header, which very briefly describes what the given chapter contains and why it is important to the project.

In the report the notation \moodlefile{/directory/file.ext} is used to specify a file in the source code. 
The path is relative in relation to the base directory of the Moodle installation. 

%\todo{M\aa{}ske noget om ``demonstration meeting'' er blevet til ``demo meeting'', check hvad der er std i scrum}
When citing literature the notation [XX] is used, where XX indicates the number of the specific literature, which can be seen in the back of the report on page \pageref{chap:bib}.
A copy of the code can be found on the attached CD. 
We refer to the source code as Appendix C.

The product is hosted on a computer by Aalborg University. 
Login information can be found on the CD.
This information is referred to as Appendix D.

We, the authors, would like to thank our supervisor Kurt N\o{}rmark for guidance, continuous support, and great enthusiasm throughout the course of the project.
We would also like to thank the people who participated in out interviews and demo meetings, which was a tremendous help.