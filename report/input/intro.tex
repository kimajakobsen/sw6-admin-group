\chapter{Introduction}
\label{chap:introProjectgroup}
Imagine an LMS that incorporates course management and Aalborg PBL model along side each other.
This is the main idea behind this project.

As of this part only the authors noted in the title page have been working actively on this report, with help from our supervisor.
Hence ``we'' only refers to the four authors of this report, and not the 14 collaborators of the entire system.

During the development of this project we use an agile development practice. 
The development practice is explained further in \chapref{sec:developmentpractice}. 
The report structure and content does not reflect the use of an agile approach and is written as it were developed using a traditional development approach. 
We structure the report in this manner to make it more readable and comprehensible. 

It should be noted that the final system, \system{}, is supposed to be a complete system, not a set of independent systems.
This means that we will be referring to the other \subsystem{}s whenever there is a dependence on them or we are making services that the other \subsystem{}s depends on.



As defined in \secref{sec:problemDef} this project is concerned with integrating the Aalborg PBL model into \moodle.
An important concept of this model is the ``Team'' (see \secref{sub:aaupbl}).
Our sub-group has the responsibility to develop a \subsystem{} that allows a team to work together through \moodle{} and some way to manage who is part of which teams with what roles. We hence call this tool ``administration tool''.



%For a team to operate in an interactive environment it is necessary to have some entity that defines the members of the team and the actions that they can perform to solve their problem.
%We call this entity a ``Project Group''.



To make \system{} usable at any educational institution we provide a tool to manage project groups -- e.g. allowing creation, modification, and archiving of project groups.
The members of the project group will have access to some shared activities through which they can conduct their group work.
These activities include planning the progress of the project, communicating internally the project is being conducted, and communicating with people not actively part of the project group, such as supervisors.
These activities are created in our peer sub-projects (as mentioned in \secref{sub:decomposingSys}).
Our responsibility is to make all these activities available to the members of the project groups.




