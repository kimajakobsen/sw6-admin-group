\section{Managing Project Groups} %Jeg skriver kun om admin funktionalitet her -Mikael ~♥
\label{sec:manProjGrpImpl}
Before any project group can be useful it has to be created first.
Administrators need to have the ability to add, edit, and delete project groups.
%The administration tools we provide have to be easy and fast to use, since they potentially have to be used many times.
This section describes how we implement the administration tools needed to manage project groups.

The features we provide to manage project groups are known as admin tools in Moodle.

The main functionality is placed in the file \moodlefile{/admin/tool/projectgroup/lib.php}, which is a code library consisting of $\approx 500$ lines of code. 
The library is know henceforth as the admin tool library.
The library includes the project group library discussed in \secref{sec:pglib}, and hereby allowing pages in the administrative part to use code in the project group library. 
This include is the implementation of the dependency between the admin tool and the project group library seen on figure \figref{fig:architecture}. 

In the following section creation, editing, and listing are presented. 
Deletion of project groups are omitted since we consider the functionality used as trivial. 
The reader can see the code for the deletion view in \moodlefile{/admin/tool/projectgroup/delete.php} and the function, \fu{delete\_projectgroup}, that does the actual deletion is located in the admin tool library.

\subsection{Add \& edit}
To add and edit project groups, the file \moodlefile{/local/projectgroup/edit.php} is loaded into the browser. 
The looks and feels of this page is described in \secref{sec:adminPrensentation}.
The same file is used for both editing and creation of new projectgroups. 
The only difference between editing a group and adding a group is that when editing a group a HTTP parameter, $id$, is set.
When $id$ is set the form fields are filled with the relevant information from the database.
Once the submit button is pressed the function \fu{save\_or\_update\_projectgroup} is called with a project group object as input.
That function checks if the $id$ of the group is set. 
If it is we know that a project group is being edited, and we can update a row in the database.
If not we know that a new project group is being created and we insert a row in the database.

\subsection{Lisiting}
In the list of project groups, mentioned in \secref{sec:adminPrensentation}, we provide a filtering functionality, which makes it easy to find a specific set of project groups.
The project group filtering is a rework of the user filtering, which already exists in Moodle.
The filter itself is placed in \moodlefile{/admin/tool/projectgroup/projectgroup\_filtering.php} and is class named \cl{projectgroup\_filtering}.
This functionality is requested by the administrative personnel (see \secref{sec:requirements}).
The project group list is rendered in \moodlefile{/admin/tool/projectgroup/index.php} and the filter is created here aswell. 
\coderef{moodlefiltering} shows the invocation of the filter. 
The filter is created and by the use of the function \fu{get\_projectgroup\_selection\_data} we get the data which we use in the creation of the projectgroup form, which handles the HTML rendering of the filter. 
\begin{lstlisting}[style=phpCode, caption=\myCaption{The invocation of the filtering mechanism}, label=moodlefiltering]
//create the user filter form
$filtering = new projectgroup_filtering();

//get the data from the form
$data = get_projectgroup_selection_data($filtering);
$data['fields'] = $filtering->get_fields();
$editform = new projectgroup_form(NULL,$data);
\end{lstlisting}\begin{comment}$\end{comment}

%\subsection{Deletion}
