\documentclass{article}

\begin{document}

\section{Introduction}
This mini project is divided into three parts: development approach, tools and techniques, and evaluation.
\subsection{The Project}
The project which this mini project is based upon is a multi project with four subgroups.
The project is to develop an extension to the e-learning system Moodle, which supports problem-based learning.
We approach the project by dividing the project into four smaller subparts.
The four subparts are: Intra group communication, planning, administration, and group-supervisor-communication.

\section{Development Approach}

\subsection{Distinguishing Factors}
\begin{enumerate}
	\item 14 persons in four groups \label{groupSize}
	\item Diverse target group \label{targetGroup}
	\item No on-site costumer \label{onsite}
	\item Hard deadline \label{deadline}
	\item Passed on project \label{passed}
	\item Known framework and platform \label{framework}
	\item Education environment \label{education}
	\item Not full-time development \label{halftime}
	\item No manager/Project owner \label{manager}
	\item No shared working room \label{room}
\end{enumerate}

\subsection{Approach \& Rationale}
We will use a development approach similar to Scrum of Scrums.
The reason for this is three fold.
First of all we have a diverse target group (point \ref{targetGroup}), that may make big up front analysis and design difficult.
This leads us to choose an agile method due to IKIWISI.

Secondly we are 14 group members divided into four groups (as point \ref{groupSize} states) which is not handled very well in other agile methods such as XP, which dictates that all the developers should be in the same room. 
This is not the case for us as point \ref{room} states.

The third reason is that we have a hard deadline (point \ref{deadline}), which means that we have to hand in our project at a specific date.
Scrum of Scrums suggests that iterations (sprints) are time-boxed, which is ideal for us since we can cut less important features instead of missing the deadline.
This is also supported by point \ref{passed}, because the end product is a working release although some features may have been cut.
The features cut may then be suggested to the group which is to take up this project next year.



\subsubsection{Diversion from Scrum of Scrums}
As point \ref{manager} and \ref{onsite} states, we have neither a project manager nor an available on-site customer.
We will handle the missing on-site customer by having shorter iterations and contacting the customers whenever an iteration is over.
Scrum of Scrums dictates that there should be a Scrum master in each subgroup.
Since none of us has used Scrum before, none of us are qualified to be Scrum master/project manager.
We are in an educational situation (see point \ref{education}) so we will strive to allow every member to try to be Scrum master for a shorter time period.
This may not be ideal, but we consider it to be more important that every member of the subgroups tries to have the responsibility of a Scrum master than having only one member trying it and learning it well.





\section{Tools and Techniques}
\end{document}
