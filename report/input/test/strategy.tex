\section{Strategy}
\label{sec:strategy}
Our development is to some extent test driven.
As mentioned in \secref{sub:testing} we are using SimpleTest to test our PHP code.
When we know what a function is supposed to do prior to its implementation we write test cases for the function.
If we do not write test cases prior to its implementation we write test cases for the function after its implementation.
Even though it is likely that the person who made the function is the same person as the one who writes the test cases, there are still bugs to be found by this method.
Since working with Moodle is new to us we do not have a full understanding of the Moodle core API, which means that in some cases we will expect a function to return something other than what it actually returns.
Many of these kind of bugs are discovered by writing test cases.

The test cases we write do not catch all bugs in the system, but that does not mean that they do not serve a purpose.
Apart from the bugs they actually catch, they also improve the robustness of the system.
If someone decides that a function should be altered the test cases for the function ensure that it still behaves as it should.
This especially important in our case since a new group of people will continue our work later.
