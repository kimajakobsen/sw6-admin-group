\section{Inter Group Collaboration}
\label{sec:intergroup}
In this section the chosen development approach and its utilization is discussed. 
\todo{Der mangler lidt intro} 
%skriv mere
%skriv noget om hvor mange we er. (14)

%Initial planning
\subsection{Initial period}
In the beginning and prior to the employment of a development method we conducted whole team assemblies. 
This practice worked well in the initial part of the project, but they became ineffective as progress went on. 
A reason for this is that the topics discussed early in the development are important and interesting knowledge to everyone, but later some topics became of no interest to some groups while important to others.
That made the enthusiasm at the big assemblies fall drastically and we organized fewer meetings.


% Choice of scrum of scrum
\subsection{Choosing Scrum of Scrum}
In one of the whole team assemblies we agreed upon that due to the high uncertainty of this project we would use a I Know It When I See It (IKIWISI) approach. 
Furthermore, we used Barry Boehm and Richard Turn's five factors to determine if we should choose a agile or traditional development approach. 
Personnel, Dynamism, Culture, and Criticality are determined on our collective opinion. 
We will now evaluate on these.    
20 \% of our whole project team are able to follow a development approach, 35 \% of our whole team are able to adapt or develop development practices. 
The last 55 \% fall below these standards.
We base this on the whole team followed the course Software Engineering~\cite[S. 4.4.2]{sw6studieordning} where we learned about development methods and the usage thereof. 
This course did not make us experts but we gained a basic understanding of how to adapt a development method, based on this we derived the mentioned percentage distribution.  
We expected that 30 \% of our initial goal would change.
This is hard to evaluate due to the fact that there where no product owner to determine a initial set of goals.
We conducted the development agile and expected a high rate of change, however it is impossible to predict if it would still be the case if we had chosen a traditional approach.  
By voting we determined that 70 \% of us prefer a chaotic contra a structured working environment.
We deemed that if the system we where to build crashed, it would only effect the comfort of the users, this still holds.

Based on these five factors we choose to use the \scrum{} methods called \sos{}.  
We chose this over XP and traditional based methods.
It is difficult to validate that the choice is correct and that \sos{} is the best suited development method compared to the others.
On the other hand; nothing indicates that the other methods should be better suited for our project. 
In the implementation of \sos{} we chose to have \sos{} meetings nearly twice a week.
This worked well since they helped bring forth some issues and mitigated many potential risks of misunderstanding between \subgroup{}s.



%Sos meetings
%%length / what we talked about
%%rate
\subsection{Scrum of Scrum meetings}
As mentioned we conducted \sos{} meeting approximately twice a week. 
At times twice a week was insufficient and at other times this was sufficient.
It worked well that the meeting was planned on a as need basis.
The \sos{} meetings have also been subjects to use for evaluation of the development method and its practices, this has lead to violation of the time constraint of the \sos{} meetings in some situations.


%sprints
\subsection{Sprints}
During our project we have had four sprints, the first one being a non-programming sprint.
The length of the sprints had a varying length, but they where always time boxed. 
The varying length of the sprint was a necessary due to the fact our course schedule was unevenly distributed. 
The number of sprints are limited to four because of the hard deadline where we are to hand in this report and the pediod that we deemed that we needed to write this report.


%Roles
%%Product owner
%%Scrum master
\subsection{Product Owner and Scrum Master}
An issue in the implementation of \sos{} is the lack of a product owner, which is described as key concept. 
We compensated for this lack by doing several interviews with possible end users and to have them included in the demo meetings. 
In effect we have had the role of the product owner our selves, which is not an optimal solution. 
%It is not optimal for the reason that it is recognized as bad \scrum{} practice not to have a product owner to prioritize tasks \cite[p.~128]{Larman04}. 
It is not optimal for the reason that it is recognized as bad \scrum{} practice to have many product owners to prioritize and select tasks for a sprint~\cite[p.~128]{Larman04}. 
Furthermore, we have conflicting interests -- we want to implement something exciting and challenging, while the end users may in fact need something we consider to be simple and boring.
Another issue is that we, the developers, had to act as a scrum master. 
This is a problem because we do not have the intended expertise to act as a scrum master.

\subsection{Sprint, Demo, and Retrospective Meetings}
%Sprint planning meetings and retrospektive meetings and demo
Before the first sprint we had a collaborate sprint planning meeting where all tasks for every \subgroup{} was presented and estimated in collaboration.
In successive sprints we decided that every \subgroup{} was responsible for handling these meetings themselves, because there was only a limited gain from sitting in on the selection and estimation of other \subgroup{}s sprint items compared to the gain from \sos{} meetings that we conducted during the sprint.

Ideally we would have conducted our demo meetings with end users immediately after a sprint ended, but in some cases we had to wait because the end users we have been in contact with was not available in the given time period.
This resulted in some feature requests that came in during a sprint and therefore had to be delayed -- we prohibited ourselves from adding work to our sprint backlog during a sprint as \scrum{} suggests.

We never conducted any formal sprint retrospective meetings in collaboration with the other \subgroup{}s.
We compensated for the missing meetings by evaluating the development method along the way. %skriv at vi gjorde det under sos møder



%Tools
%%version control
%%bug tracking
%%code documentation
%%Testing
%We choose to use Hg as a mean to have version control on our code. 


%collaboration in general
%%Not shared group room
\subsection{Group Room Location}
Another issue is the lack of a shared room, which is preferred in \sos{}, but it helped that the group rooms are located close to each other and we quickly discovered that letting the door stand open during work hours helped on the inter group communication. 

In general the development approach worked well and it made us adapt to changes. 
An example of a large change during development was when we realized that it was impossible to use activity modules in project groups without linking them through a course.
This was overcome by changing the already implemented activity modules into \block[]s. 
The changed requirement was embraced and quickly implemented.
To make this change quickly and effective it helped a great deal that our group rooms are located close to each other, since we could communicate with the other project groups as soon as we found the solution.
The solution was found in collaboration with some of our peer-groups.





