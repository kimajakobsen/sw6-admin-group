\section{Requirements}
\label{sec:requirements}
During this project we have had several meetings with personnel that we consider to be our end users.
Some of them before coding the system and some of them during coding.
Here we present the final requirements.
The continues change in our requirements is evaluated in \secref{sec:intergroup} in \partref{part:evaluation}.
The interviews that have been conducted to gather these requirements are found in \appref{app:interview}.

Since we use scrum as our main development practice we use a product backlog to capture our requirements~\cite[p.~114]{Larman04}.
Different sources suggest different notations for what is in a backlog~\cite[p.~17]{scrumchecklist}\cite[pp.~123-124]{Larman04}.
Our backlog items can be a use cases, features, or fixes.
Since this project is to be proceeded by a new group of students next year, we allow our product backlog to be larger than we are able to complete in this project.
We will now focus on the items that are part of this semester project.
These are said to be part of our release backlog.
The backlog items that makes up our product backlog for our \subsystem{} are defined in \figref{fig:productbacklog}.
The process' behind creation the most central backlog items are also presented here.

%skriv flere cases af hvordan b-items er opstået 
In the interview described in \appref{sec:lene} we learned that there is a need for having a concept of project groups in \moodle{} to enhance communication by sending messages from the \admpers{} to project groups through \moodle{}.
This evolves into two backlog items; integrate the concept of project groups into \moodle{} and allow the administrative personnel to post messages to the project group.
The former is represented as a backlog item which can be seen in \figref{fig:productbacklog}.
The latter is implemented by the \supervisorgroup{}.

\begin{figure}%
\begin{tabular}{|p{0.2\textwidth}|p{0.1\textwidth}|p{0.7\textwidth}|}
	\hline
	
	\textbf{Name} & \textbf{Priority} & \textbf{Description} \\
	\hline
	\textbf{Manage Project Groups} & \textbf{Priority} & \textbf{A subset of our target group should be able to manage the project groups, this include create, edit, and delete project groups. } \\
	\hline
	\textbf{Virtual Group Room} & \textbf{Priority} & \textbf{Create a virtual space that are compliant with the principals of the Aalborg PBL model and create a feeling of nearness. } \\
	\hline
	\textbf{Recursive Project Groups} & \textbf{Priority} & \textbf{Allow project groups to contain groups of persons. This also include the administrative tools to manage the groups and sub-groups.} \\
	\hline
	\textbf{Introduce Roles} & \textbf{Priority} & \textbf{Add the possibility to assign roles to the members of a project group, roles such as supervisor and student.  } \\
	\hline
	\textbf{Manage Roles} & \textbf{Priority} & \textbf{Create Functionality to define and manages roles. An example of a new role  could be secondary supervisor} \\
	\hline
	\textbf{Find Project Groups} & \textbf{Priority} & \textbf{Produce functionality to find a given project group. This involves creating a list of project groups and making features to allow search or filtering.} \\
	\hline

	\textbf{Navigate to Project Groups} & \textbf{Priority} & \textbf{The members of the project group should have a way to navigate to it.} \\
	\hline
	\textbf{Project Group Members} & \textbf{Priority} & \textbf{It should be possible to see who the members of the group is, to create the feeling of nearness.} \\
	\hline
	\textbf{Project Group Synchronization} & \textbf{Priority} & \textbf{Project groups should reflect the physical project groups at the university. Thus when a student joins an existing group, the student should automatically be added to the virtual group room. } \\
	\hline 


	
	
	
\end{tabular}
\morscaption{A list showing our product backlog. The italic items are not part of our release backlog}
\label{fig:productbacklog}
\end{figure}

%Få skrevet noget og hvem der skal gøre hvad.
