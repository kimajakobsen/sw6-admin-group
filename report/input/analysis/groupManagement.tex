\section{Project Group Management}
\label{sec:groupManagement}
\label{sec:projectgroupadministration}
%From \secref{sec:virtualMeetingPlace} we know that we need a notion of projects or project groups.
%In either case instances projects or project groups must be created in the system in some way, as the requirement Manage Project Groups states in \secref{sec:requirements}.
%Other requirements considered in this section are: Find Project Group and Introduce Roles.
%\todo{I det her kapitel er der anvendt project group når der tales om Moodle og groups eller student groups når der tales IRL, %Sørg for det giver mening når du læser det igennem ellers så ret det.}
%\subsection{The Approaches} %% Very much working title
% Creation % Editing (Change of name and similar, add members) % Deletion
During the meetings with ELSA,  see \secref{sub:elsaInterview}, and the \admpers{} we discovered three different approaches to administrate the project groups: Automatic management through external application, manual management by administrative personnel, and administration by students. 

\subsection{Automatic Management}
\label{sub:automanagement}
%% Automatic Creation(Proposed by ELSA) (Michael)
In this section an automatic group management approach is explored.  

%ELSA proposes that groups should be created automatically through an external system. 
Courses are created automatically on Moodle by ELSA.
%This technique is already used at Aalborg University to create courses. 
The courses are not populated automatically, but it still eases the work load for the \admpers{}.  
A similar approach can be used to create groups from an external database containing the student groups. 

With automatic management through an external system it is necessary to synchronize Moodle with the external system in the case that the groups are changed. 
If the data is inputted only in the external system a one-directional synchronization is needed, but if groups can be created, added, and deleted in Moodle a bidirectional synchronization service must be created. 
This service must be able to handle conflicts if changes are done at the external application and Moodle at the same time.

After an interview with Michael Møller Hansen of IST (see \appref{sec:mikael}) we discovered that AAU does not have a single system containing the groups at the University.
Several systems exist and some of the systems are not online.
Each of these systems contain a fragment of the entire set of project groups.
Furthermore, some departments does not keep official records of the groups (see \appref{sec:mikael}).
Morten Mathiasen Andersen explained in an interview that the systems do not have an API from which the groups can be extracted and he himself is working on an API for one of the systems.
%Without an API from which to extract the groups this approac. 



%%% Maintanance 


\subsection{Student Management}
%% User- self-Creation  (Else talks about maintanance should be done by students)
In this section an approach which lets students create, edit, and delete groups is discussed. 

In some cases the administrative personnel do not know which student groups are created (see \appref{sec:mikael}), and it is not guaranteed that all students want an online project group in Moodle. 
Therefore, a solution is to allow students to administrate, allowing the students to create and edit their project groups as needed without adding extra workload to the administrative personnel.

The problem of this approach is that students might create badly named groups or misuse the system by e.g. destroying groups for other students. 
This approach must include various security constraints to ensure proper use. 
Students should not be able to delete groups they are not members of or to delete groups by mistake. 
A mistaken deletion of a group late in the semester could lead to a large amount of lost information and work. 
The creator of the project group must be able to govern who can be a member or the members of the group must accept new members.
Alternatively they should not be able to add or remove members from a group after creation since a student should not be able to join a group they do not belong to. 


\subsection{Administrative Personnel Management}
%% Administration personel creation (Lene creates courses and notes who is in the groups)
In this section the approach of having the administrative personnel administrate project groups is discussed. 

The administrative personnel plays a central role in the administration of the courses and student groups and is thereby capable of adding the necessary project groups to Moodle. 
Since the administrative personnel is trained professionals this approach ensures that only the needed project groups are created and that they are maintained properly. 
%That is with good naming convention.
A drawback of this approach is that it adds more work to the administrative personnel.

%\ \\ 

%----------------------------------------------------7\subsection{Design}
\subsection{Choosing an Approach}
The automatic management approach using an external system can be ruled out due to the fact that there does not exist a central student group database at AAU and the decentralized system, AdmDB, in the process of getting a new API developed.
Thus are the current API gonna be deprecated in a near future. 
If a centralized system with an API to extract group information existed this would be an the ideal approach, since it will not add workload to the administrative personnel and all groups will be able to have a project group in Moodle. 
From the two remaining approaches we choose to have the administrative personnel administrate the project groups. 
We choose this because the student administration approach has several constraints which must be overcome. 
With the student approach the administrative personnel must be able to clean up the project groups and therefore an administration panel for the \admpers{} is necessary.
Without having to spend time figuring out how to constrain the system to prevent misuse we can implement a working system by choosing the administrative personnel management approach.
The two approaches do not exclude each other and the functionality used in the chosen approach can be reused if the other approach is implemented later.

With the chosen approach administrators need to be able to add, remove, and edit project groups.
Per default Moodle has a \block{} called Settings. 
If a user is an administrator there is a list in this block called site administration. 
This list contains a lot of tools administrators use, and we choose to add project group administration to this list. 

We want a page where administrators can add a new project group that also adheres to Moodle standards.
It should be possible to name a group, add the desired members, and choose who the supervisors of the project groups are.

Additionally there needs to be a list of all project groups.
From this list it should be possible to add, edit, and delete groups. 
This list can potentiality grow very large, so there needs a way of finding a specific group or a specific set of groups as specified by requirement Find Project Group.
This requirement is fulfilled by developing a filter system where an administrative user can add filters on attributes to find a specific project group or a set of related project groups.
The filter attributes should be those that administrative users are likely to use.
During the demo meeting described in \appref{sec:lenedemoone} we experienced that our end user need to find project group based on its name.
This has lead us to decide that the project group name should be a filter attribute.

The final requirement to consider is Introduce Roles.
An administrative user must be able to set roles of the members of a project group.
We choose to allow an administrative user to change the role of a project group member in the page where he can add and remove members.
The roles that we consider to be important in a project group are student and supervisor.
An option we consider is to allow administrative personnel to create roles themselves, to adapt to a given situation.
However, we have not found any need for this, hence we choose only to have the two roles student and supervisor.





