\section{Requirements}
\label{sec:requirements}
During this project we have had several meetings with personnel that we consider to be our end users.
Some of them before coding the system and some of them during coding.
Here we present the final requirements.
The continues change in our requirements is evaluated in \secref{some_label} in \partref{part:evaluation}.
The interviews that have been conducted to gather these requirements are found in \appref{app:interview}.

Since we are using scrum as our main development practice we have been using a product backlog to capture our requirements~\cite[p.~114]{Larman04}.
Different sources suggest different notations for what is in a backlog~\cite[p.~17]{scrumchecklist}\cite[pp.~123-124]{Larman04}.
Our backlog items can be a use cases, features, or fixes.
Since this project is to be proceeded by a new group of students next year, we allow our product backlog to be larger than we are able to complete.
We focus on the items that are part of this semester project.
These are said to be part of our release backlog.
The backlog items that makes up our product backlog for our \subsystem{} are defined here.
The process' behind the most central backlog items are also presented here.

%å
In the interview described in \appref{sec:lene} we learned that there is a need for having a concept of project groups in \moodle{} to enhance communication by sending messages from the \admpers{} to project groups through \moodle{}.
This evolves into two backlog items; integrate the concept of project groups into \moodle{} and allow the administrative personnel to post messages to the project group.
The former is represented as a backlog item which can be seen in \figref{}.
The latter is implemented by the \supervisorgroup{}.

\begin{figure}%
\begin{tabular}{|p{0.2\textwidth}|p{0.2\textwidth}|p{0.6\textwidth}|}
	\hline
	\textbf{Name} & \textbf{Priority} & \textbf{Description} \\
	\hline
	
\end{tabular}
\morscaption{A list showing our product backlog. The emphasized items are not part of our release backlog}
\label{fig:productbacklog}
\end{figure}
