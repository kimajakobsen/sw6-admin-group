\chapter{Interviews}

\section{Lene Winther Even}
\label{sec:lene}
Interviewers: Kim A. Jakobsen and Mikael Midtgaard\\


This interview was conducted on March 15\ths{} at Cassiopeia at Aablorg University.
At the time of writing Lene is a Senior Secretary at Aalborg University, and one of her tasks is to handle many aspects of Moodle for the Institute of Computer Science.

The following is a summery of the interview.

\subsection*{Moodle in General}
It is a problem that there are 13 different Moodle systems at Aalborg University, especially since some student are required to use more than one Moodle system.

When there is a change in a calender event it is difficult to administer in Moodle.

Only students, administrative personnel, teachers, supervisors use Moodle.
Research groups do not use Moodle to share information.

\subsection*{Project Groups}
Project groups exists in a document written manually in a course.
Currently the university do not differ between groups and projects.
Messages for project groups could be improved by sending the messages through Moodle instead of email.
Messages to entire semesters are already sent through Moodle.
A forum can be created for each project group to be used for a group's internal and supervisor communication.
%The implementation in Moodle version 1.9 by default sends emails to the whole semester when .

\subsection*{Other Tools}
Lene tells us of some of the other tools that are used in relation to the Moodle system.

\subsubsection*{ADMDB}
ADMDB is the Administration Database and is maintained by the Information Services Technology department(IST).
The database is not currently linked with Moodle; all information from the database has to be written manually into the Moodle system.
The database was linked with the old TYPO3 system.
%We learned more about ADMDB from SOMEGUY ALSO MAKE REF. \todo{ref to other interview}

\subsubsection*{CalMoodle}
CalMoodle is a calender system for courses.
Calender events are created in CalMoodle and imported to the Moodle system.
Calender events have no direct relation to courses.

\subsubsection*{Office}
Different programs from the Microsoft Office suite are used to store and manage data locally.

\subsection*{Courses}
Some courses at Aalborg University share some of the same information.
However, the common information has to be written to the courses individually.
It is highly wanted for courses to share some common information.

Administrative personnel are enrolled to many courses, which all are listed on the front page and in the navigation menu.
This clutters the main page and makes navigating to a specific course difficult.

School of Information and Communication Technology(SICT)\todo{kontroller at vi ikke har brugt denne forkortelse tidligere} are responsible for creating a Moodle course page for each course as well as a Moodle course page for each semester. 
%Students are automatically assigned to the semester course, this is done by STADS.
The administrative personnel write some general information to the courses.
Additional information are written by the lecturer.
Course information is often copied and pasted different places.

Students are automatically added to the semester course, but they have to manually enroll to every other course.
Teachers are added to courses with the role of a student, and an administrator has to manually promote them.

\subsection*{Archiving}
Finished reports are the only thing that is currently being archived.
They are archived in the project database.
Supervisor contact and meetings are relevant candidates for being archived.
ELSA have plans for archiving all courses and course information for future reference.