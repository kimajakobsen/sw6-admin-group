\subsection{Ensuring Permissions}
\label{sec:projectgrouproommanagerights}
Permissions can generally be divided in two types; read and write. 
Read permissions gives you the ability to view the content of the virtual group room while write lets you change the content. 
If a user has write permission he must also have read permissions, otherwise he cannot see the page he attempts to edit. 
To ensure the user attempting to enter a virtual group room has permission to enter the function \fu{has\_projectgroup\_read\_permission} is used. 
%It checks if the user is a member of the group. 
It checks if the user is an administrator or is a member of the project group. 
The administrator check is necessary since an administrator should be able to see any project group even if he is not a member of the project group. 

The function \fu{has\_projectgroup\_write\_permission} checks that the user has write permissions and uses the read permission function to check that the user has read permission.
If he does not have read permission he should not be able to edit the virtual group room. 
In the current implementation the write permissions function does not make extra checks to permissions since the permission levels for read and write are equivalent.

Because the functions are separate it is possible to change them individually later.
%Making both function gives the ability to later change this.
An example is that an end users requires that supervisors should only have read permissions. Then the change will only be in one place. 
Both functions can be found in the project group library.
