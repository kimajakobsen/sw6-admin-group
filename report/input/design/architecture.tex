\section{Architecture}

To implment MyMoodle a lot of different moodle plugins are needed. 
MyMoodle can be seen as a plugin of plugins. 

(Argumentation for an architecture)


The architecture does not specify how exactly which plugins should be created but specify a general structuring of the parts of the system. 


The complete architecture can be seen in figure JADAJADA
The box surrounding the parts illustrates a common dependency of the moodle platform. 
Every part, except for the database, is dependent on Moodle to work. 
The system consist of a total of five layers. 
The three uppermost is implemented in this project while the lowest two are existing prior to the project. 
The database is extended but otherwise are theese two layers unchanged.
The upper most layer is the projectgroup view and the administration tool.
The project group view is used for presenting the projectgroup room described in JADA JADA and the administration tool is a tool used by managing personal for creating, editing, and deleting groups. 
Directly below the upper most layer is the ``middle'' layer which consist of the four parts: THE FOUR PARTS. 
EXPLAIN THF FOUR PARTS
Below the middle layer is the project group library which contain common functionality and can be used by everyone.  



There are two primary factors for creating an layered architecture. One is that we are four groups working together, which creates the need for a structured way of dependency amoung the groups. Two is that the project should be passed on and a with a good architecture the comprehensibility of the project is increased. 
It is not possible to make a strict layered architecture due to the moodle dependency and the admin which does not have a intermediate layer, but depends directly on the project group library












disp: 

Why, what, why







