\chapter{Future Work}
\label{chap:futurework}
\myTop{In this chapter we present the ideas that we believe to be the most valuable additions to \system{}.
While \chapref{chap:evalProduct} focuses on alterations and improvements to the system, this chapter focus on new features and alternative usages of the system. There are several features and extensions to the system, but only those we regard as highest prioritized are mentioned. 
This chapter is relevant because students should continue on our work later. }

%%FYSISKE RUM -> forståelse -> booking

\section{Physical Group Rooms}
\label{sec:booking}
In the current implementation of \system{} there is a concept of group rooms.
This concept is strictly virtual -- there is no understanding of actual physical group rooms.
%If a group has the same physical group room during an entire semester, it would also be possible to identify the group by their group room.

If there were an implementation of physical group rooms in Moodle it would be possible to implement a room booking system.
As mentioned in \secref{sec:virtualMeetingPlace} some groups have to book group rooms daily.
This takes the time of \admpers{} and students.
Such a system could also be expanded to handle the booking of lecture rooms.

There are several constraints that have to be considered before implementing such a system.
First of all there should be priorities based on the person wanting to reserve a room, e.g.\ a lecturer should take precedence over a student when trying to reserve a lecture room.
Additionally there should be a system in place that handles a situation where there are not a sufficient number of physical group rooms available; no project groups should be able to make a significantly greater number of reservations than other project groups.

We believe such a system has a sufficient complexity and relevance that it could serve as a project for one of the $6$\ths~semester software groups of next year.

\section{Central Project Group Database}
An issue which is discussed in \secref{sec:evalAdministration} is the lack of a single system for managing all project groups at Aalborg University. 
This issue can be solved by letting Moodle become the authoritative database for project groups. 
To become this several issues must be overcome. 

\begin{itemize}
	\item The structure of the project groups must be very general to support all the possible project group structures that exist at Aalborg University. 
	\item There should be only one Moodle installation with one database system. 
	\item The system must have a strong backup system.
	%\item The current system should also have backup, but as a primary database for crucial information it is even more important. 
	\item The system should have an accessible API from which project groups can be exported to other systems if needed. 
\end{itemize}

%The implementation issues concerned with Moodle becoming a primary database is minor compared to the task of getting the administrative board at Aalborg University to decide to implement and use the system. 
We do not believe that this is a valuable feature to implement as long as there exist more than one Moodle installation at Aalborg University. 

%\section{User Self Management}

\section{Virtual Meeting Place Templates} 
\label{sec:templates}
%In \secref{sec:releaseBacklog} the featue called Virtual Meeting Place Template is discussed. 
Templates for virtual meeting places will enable administrative personnel to create virtual group rooms with predefined \detdeandrelaver{}s.
In the current system we have one template for the virtual meeting place. 
This template specifies a set of activated blocks for the virtual meeting place. 
These blocks have been chosen by us, but our choice for this may not be optimal for all departments and by having multiple templates each department can have their own template.

%To implement this feature the function \fu{blocks\_add\_default\_projectgroup\_blocks} must be changed so it instead of getting the list of default blocks from a configuration it should get the name of a list.
To implement this feature the function \fu{blocks\_add\_default\_projectgroup\_blocks} must be changed so it instead of getting a list of default blocks from a configuration file should get a name of a template.
Based on this name the specific set of \detdeandrelaver{}s should be added to the virtual group room.
%it should fetch the name of a list which is then fetched and parsed. 

To figure out which blocks are best suited as default for the virtual meeting places in the various departments the students and supervisors at the departments must be interviewed or answer questionnaires.

Alternatively, the students could be granted privileges to create their own templates and share these with each other.
This way we avoid the risk of having conflicting wishes about templates in a given department.

Implementing this feature may not constitute for an entire semester project, but is never the less a feature that can be considered.
Perhaps it can be implemented as part of a semester project.