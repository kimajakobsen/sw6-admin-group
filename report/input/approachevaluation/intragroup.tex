\section{Intra Group Development}
\label{sec:intragroupdev}
%what will this section be about.

In this section the most relevant aspects of our development approaches are evaluated.

\subsection{Scrum Roles}
%The development method we used in this project is \scrum{}.
We were unable to follow the \scrum{} development method completely because some roles were not filled.
While still a problem, fulfilling all roles is not paramount in the intra group development of this project.
We had an interchangeable \scrummaster{}; we took turns taking the role of \scrummaster{}.
This worked sufficiently well, but it would be preferable to have the role filled by an experienced \scrummaster{}.
One could argue that it would have been preferred to have only one person take the role of \scrummaster{}, since he would gain more experience and the development would be more uniform.
However, since this is a learning project we all would benefit from being the \scrummaster{} even from a single sprint.

An issue in the implementation of \sos{} is the lack of a product owner, which is described as a key concept. 
We compensated for this lack by doing several interviews with possible end users and to have them included in the demo meetings. 
In effect we have had the role of the product owner ourselves, which is not an optimal solution. 
%It is not optimal for the reason that it is recognized as bad \scrum{} practice not to have a product owner to prioritize tasks \cite[p.~128]{Larman04}. 
It is not optimal for the reason that it is recognized as bad \scrum{} practice to have many product owners to prioritize and select tasks for a sprint~\cite[p.~128]{Larman04}. 
Furthermore, we have conflicting interests -- we as programmers want to implement something exciting and challenging, while the end users may in fact need something we consider to be simple and boring.

%skriv om hvordan vores planning poker har udviklet sig. 
%skriv om hvordan story points are udviklet sig. og hvordan vi har lært at bruge dem. FYI så er story points ikke forklaet i rapporten, men kun nævnt.


\subsection{Management of Requirements}
When planning a sprint we used planning poker, which is a \scrum{} practice to estimate how much time a task takes. 
%This method got progressively better, because
In the beginning we had very different ideas of how much time a story point was and we did not decide anything specific. 
After working on tasks our valuation of story points became increasingly similar.
%backlog
%The backlog items we valuated by planning poker were split into smaller tasks as work began on them.

%% Taks
We tried several different approaches for estimating how many story points the size of a backlog item corresponds to. 
%In the first sprint we decided which items we approximated what we could do in a sprint planning meeting with our peer-groups.
In the first sprint we estimated our sprint log items in collaboration with our peer groups as described in \secref{sub:interdevmeet}.

In the second sprint we chose a few items from our release backlog and after estimating them using planing poker we decomposed the items into smaller tasks, which were then estimated. 
This lead to a large number of small tasks, but as progress went on we realized that some tasks were redundant and could be combined. 
%In other cases there were not enough tasks to implement a backlog item. 
In other cases a few tasks were missing to fully implement a backlog item.

In the third sprint we tried to estimate only the backlog items.
This made it difficult to delegate small tasks and estimate the number of story points burned.

In the fourth sprint we created one task for each backlog item and the person who took a task had the job of starting to program and create smaller tasks that other people could take. 
This proved to be the most efficient method of distributing tasks.

A burndown chart of remaining story points was a part of the tasks management.
It was positioned in a visible position and also served as a motivational tool.

We continued to use a burndown chart when we transferred from mainly programming to mainly writing report.
After the transition we just had report tasks such as ``Write test chapter''.
These tasks were given a story point value (usually just 1) and the burndown chart showed the progress by having the story points drop or rise from day to day.

\subsection{End User Representatives}
During the development we relied on several end user representatives in the gathering of our requirements.
The selected representatives were not as spread out as we initially wished for. 
This is mainly because it is difficult to find end users who are willing to volunteer for interviews. 
For both categories of end users -- the users of the virtual group rooms and the managers of project groups -- we lack end user representatives from the Faculty of Medicine.
For the usage of the virtual group room we need users with high experience and to interview more than one supervisor.
The fact that we did not fully cover all types of people for interviews does not render the requirements useless, but it gives room for further improvement.