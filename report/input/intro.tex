\chapter{Introduction}
\label{chap:introProjectgroup}
As of this chapter ``we'' only refers to the four authors of this report, and not the 14 collaborators of this project.

As defined in \secref{sec:problemDef} this project is concerned with integrating the Aalborg PBL model into Moodle.
An important concept of this model is the ``Team'' (see \secref{sub:aaupbl}).
For a team to operate in an interactive environment it is necessary to have some entity that defines the members of the team and the actions that they can perform to solve their problem.
We call this entity a ``Project Group''.

To make \system{} usable at any educational institution we provide some form of administrative tool to manage project groups -- e.g. allowing  creation, modification, and archiving of project groups.
The members of the project group will have access to some shared activities through which they can conduct they group work.
These activities include planning the course of the project, communicating internally during the course of the project, and communicating with people not actively part of the project group, such as supervisors (se mentioned \secref{sub:decomposingSys}).
These activities are created in our peer sub-projects.
Our responsibility is to make all these activities available to the members of the project groups.
