\chapter{Future Work}
\label{chap:futurework}
\myTop{In this chapter we present the ideas that we believe to be the most valuable additions to \system{}.
Whereas \chapref{chap:evalProduct} focuses on alterations and improvements to the system, this chapter focus on new features and alternative usage of the system. There are several features and extensions to the system, but only those we regard as highest prioritized are mentioned. 
This chapter is important since the work should be continued later. }

%%FYSISKE RUM -> forståelse -> booking

\section{Physical Group Rooms}
\label{sec:booking}
In the current implementation of \system{} there is a concept of group rooms.
This concept is strictly virtual -- there is no understanding of actual physical group rooms.
%If a group has the same physical group room during an entire semester, it would also be possible to identify the group by their group room.

If there were an implementation of physical group rooms in Moodle it would be possible to implement a room booking system.
As mentioned in \secref{sec:virtualMeetingPlace} some groups have to book group rooms daily.
This takes the time of administrators and students.
Such a system could also be expanded to handle the booking of lecture rooms.

There are several constraints that have to be considered before implementing such a system.
First of all there should be priorities based on the person wanting to reserve a room, e.g. a lecturer should take precedence over a student when trying reserve a lecture room.
Additionally there should be a system in place that handles a situation where there are not a sufficient number of group rooms available; some groups should not be able to make a significantly greater number of reservations than some other groups.

We believe such a system has a sufficient complexity and relevance that it could serve as a project for one of next year's $6$\ths~semester software groups.

\section{Central Project Group Database}
An issue which is discussed in \secref{sec:evalAdministration} is the lack of a single system for managing all project groups at Aalborg University. 
This issue can be overcome by letting Moodle become the authoritative database for project groups. 
To become this several issues must be overcome. 
The structure of the project groups must be very general to support all the possible project groups structures that exist at Aalborg University. 
There should be only one single Moodle installation with one database. 
The system must have a strong backup.
The current system should also have backup, but as a primary database for crucial information it is even more important. 
The system should have an accessible API from which project groups can be exported to other systems if needed. 

The implementation issues concerned with Moodle becoming a primary database is minor compared to the task of getting the administrative board at Aalborg University to decide to implement and use the system. 
We do not recommend this as a feature valuable of implementation as long as there exists more than one Moodle installation at Aalborg University. 

%\section{User Self Management}

\section{Virtual Meeting Place Templates} 
\todo{skriv om at bruge templates til at oprette project group og hvordan det kunne gøres. Dette emner er berørt i requirements.}
%In \secref{sec:releaseBacklog} the featue called Virtual Meeting Place Template is discussed. 
The feature will enable administrative personnel to create project groups predefined templates for their virtual meeting place. 
In the current system we have one template for the virtual meeting place. 
The template specify a set of activated blocks for the virtual meeting place. 
These blocks have been chosen by us, but our choice for this might not be optimal for all departments and by having multiple templates each department can have their own template.

To implement this feature the function \fu{blocks\_add\_default\_projectgroup\_blocks} must be changed so it instead of getting the list of default blocks from a configuration it should fetch the name of a list which is then fetched and parsed. 

To figure out which blocks are best suited as default the virtual meeting places in the various departments the students and supervisors at the departments must be interviewed or questionnaires can be used. 