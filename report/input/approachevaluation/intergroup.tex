\section{Inter Group Collaboration}
\label{sec:intergroup}
In this section the chosen development method and its utilization is discussed. 
%\todo{Der mangler lidt intro} 
%skriv mere
%skriv noget om hvor mange we er. (14)

%Initial planning
\subsection{Initial Period}
Prior to choosing a development method we conducted whole team assemblies. 
This practice worked well in the initial part of the project, but they became ineffective later on. 
A reason for this is that the topics discussed early in the development were important and interesting knowledge to everyone, but later some topics were uninteresting for some \subgroup{}s while important to others.
That made the enthusiasm at the whole team assemblies fall drastically and we organized fewer of these meetings as a result.


%%%%%%%%%%%%%%%%%%%%%%%%%%%%%%%%%%%%%%%%%%%%%%%%%%%%%%%%%
\begin{comment}
%%%%%%%%%%%%%%%%%%%%%%%%%%%%%%%%%%%%%%%%%%%%%%%%%%%%%%%%%
% Choice of scrum of scrum
\subsection{Choosing Scrum of Scrum}
In one of the whole team assemblies we agreed upon that due to the high uncertainty of this project we would use a I Know It When I See It (IKIWISI) approach. 
Furthermore, we used Barry Boehm and Richard Turn's five factors to determine if we should choose an agile or traditional development approach. 
Personnel, Dynamism, Culture, and Criticality are determined on our collective opinion. 
We will now evaluate on these.    
20 \% of our whole project team are able to follow a development approach, 35 \% of our whole team are able to adapt or develop development practices. 
The last 45 \% fall below these standards.
We base this on the whole team followed the course Software Engineering~\cite[S. 4.4.2]{sw6studieordning} where we learned about development methods and the usage thereof. 
This course did not make us experts but we gained a basic understanding of how to adapt a development method, based on this we derived the mentioned percentage distribution.  
We expected that 30 \% of our initial goal would change.
This is hard to evaluate due to the fact that there where no product owner to determine a initial set of goals.
We conducted the development agile and expected a high rate of change, however it is impossible to predict if it would still be the case if we had chosen a traditional approach.  
By voting we determined that 70 \% of us prefer a chaotic contra a structured working environment.
We deemed that if the system we where to build crashed, it would only effect the comfort of the users, this still holds.

Based on these five factors we choose to use the \scrum{} methods called \sos{}.  
We chose this over XP and traditional based methods.
It is difficult to validate that the choice is correct and that \sos{} is the best suited development method compared to the others.
On the other hand nothing indicates that the other methods should be better suited for our project. 
In the implementation of \sos{} we chose to have \sos{} meetings nearly twice a week.
This worked well since they helped bring forth some issues and mitigated many potential risks of misunderstanding between \subgroup{}s.
%%%%%%%%%%%%%%%%%%%%%%%%%%%%%%%%%%%%%%%%%%%%%%%%%%%%%%%%%
\end{comment}
%%%%%%%%%%%%%%%%%%%%%%%%%%%%%%%%%%%%%%%%%%%%%%%%%%%%%%%%%

\subsection{Choosing a Development Method}
In the beginning of the project we decided to use \sos{} based on the 11 deciding factors described in \secref{subsec:choosingmethod}.
%We have only used the one development method for this project, but we still believe we made the right choice based on the deciding factors.
We have not tried any other development method for a multi-project, but we still believe we made the right choice based on the deciding factors.


%sprints
\subsection{Sprints}
During our project we have had four sprints, the first one being a non-programming sprint.
The sprints had a varying lengths, but they where always time boxed. 
The varying lengths of the sprints was necessary due to the fact that our course schedule was unevenly distributed. 
The number of sprints is limited to four because of the hand in deadline of this report.




\subsection{Meetings}
\label{sub:interdevmeet}
%Sprint planning meetings and retrospektive meetings and demo
Before the first sprint started we had a collaborate sprint planning meeting where all tasks for every \subgroup{} was presented and estimated in collaboration.
In successive sprints we decided that every \subgroup{} was responsible for handling these meetings themselves, because there was only a limited gain from sitting in on the selection and estimation of sprint items of other \subgroup{}s  compared to the gain from \sos{} meetings that we conducted during the sprint.

Ideally we would have conducted our demo meetings with end users immediately after a sprint ended, but in some cases we had to wait because the end user representatives were not available in the given time period.
This resulted in some feature requests that came in during a sprint and therefore had to be delayed -- we prohibited ourselves from adding work to our sprint backlog during a sprint as \scrum{} suggests.

We conducted \sos{} meetings approximately twice a week, however, we planned meetings on an as need basis, which worked well.
\sos{} meetings were also used to evaluate our development method, which in some cases caused the meetings to last longer than is normally allowed in \sos{}.

\scrum{} dictates that after every sprint a sprint retrospective meeting should be held to evaluate the development during the sprint in order to make the next sprint more pleasant and effective.
We never conducted any formal sprint retrospective meetings in collaboration with our peer-groupss.
We compensated for the missing meetings by evaluating the development method along the way or as part of the \sos{} meetings. 



%Tools
%%version control
%%bug tracking
%%code documentation
%%Testing
%We choose to use Hg as a mean to have version control on our code. 


%collaboration in general
%%Not shared group room
\subsection{Group Room Location}
Another issue is the lack of a shared room, which is preferred in \sos{}, but it helped that the group rooms are located close to each other. 
We quickly discovered that leaving the door open during working hours helped on the inter \subgroup{} communication. 



\subsection{Tools}
In this section we evaluate on the used tools during development. 

\paragraph{Version Control}
We chose to use HG as our version control system. 
This worked well with the structure we set up where each person and group had its own repository on the server, which allowed for constant integration. 
There were some minor problems regarding permissions, but this was fixed by changing the server configuration. 


\paragraph{Bug Tracking}
Bugzilla was installed and configured, but it was never used. 
The reason is that we did not spend time to document bugs, but fixed them immediately. 

\paragraph{Documentation}
We documented the code using code comments, which can be parsed by PHPDocumentor or PHPXref. 
PHPXref was installed on the server and used frequently. 
PHPDocumentor was installed, but we never compiled the code with it. 

\paragraph{Testing}
The use of the built-in testing framework in \moodle{} worked well and the test cases are now an integrated part of the code.


\subsection{Summary}
In general the development method worked well and it enabled us to adapt to changes. 
An example of a large change during development is when we realized that it was impossible to use activity modules in virtual group rooms without linking them through a course.
This was overcome by changing the already implemented activity modules to \block[]s. 
The changed requirement was embraced, as agile methods recommend, and quickly implemented.
To make this change quickly and effectively it helped a great deal that our physical group rooms were located close to each other, since we could communicate with our peer-groups as soon as we found the solution.
The solution was found in collaboration with some of our peer-groups.