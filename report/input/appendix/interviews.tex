\chapter{Interviews}

\section{Lene Winther Even}
Interviewers: Kim A. Jacobsen and Mikael Midtgaard\\


This interview was conducted on March 15\ths{} at Cassiopeia at Aablorg University.
At the time of writing Lene is a Senior Secretary at Aalborg University, and one of her tasks is to handle many aspects of Moodle for the Institute of Computer Science.

The following is a transcript of the interview.

\subsection{Moodle in General}
It is a problem that there are 13 different Moodle systems at Aalborg University, especially since some student are required to use more than one Moodle system.

When there is a change in a calender event it is difficult to administer in Moodle.

Only students, administrative personnel, teachers, supervisors use Moodle.
Research groups currently do not use Moodle to share information.




\subsection{Project Groups}
Project groups are created manually by writing them in a course.
Currently the university view groups and projects as one.
Messages for project groups are sent by email.
This could be improved by sending the messages through Moodle.
Messages to entire semesters are sent through Moodle.
A forum can be created for each project group to be used for a group's internal and supervisor communication.
The implementation in Moodle version 1.9 by default sends emails to the whole semester.




\subsection{Other Tools}
Lene could tell us of some of the other tools that were used in relation to the Moodle system.

\subsubsection{ADMDB}
ADMDB is the Administration Database and is maintained by the Information Services Technology department(IST).
The database is not currently linked with Moodle; all information from the database has to be written manually into the Moodle system.
The database was linked with the old TYPO3 system.
We learned more about ADMDB from SOMEGUY ALSO MAKE REF. \todo{ref to other interview}

\subsubsection{CalMoodle}
CalMoodle is a calender system for courses.
Calender events are created in CalMoodle and imported to the Moodle system.
Calender events have no direct relation to courses.

\subsubsection{Office}
Different programs from the Microsoft Office suite are used to store and manage data locally.




\subsection{Courses}
Some courses at Aalborg University share some of the same information.
However, that information has to be written to the courses individually.
It is highly wanted for courses to share some common information.

Administrative personnel are enrolled to a many courses, which all are listed on the front page and in the navigation menu.
This clutters the main page as well as making navigating to a specific course difficult.

School of Information and Communication Technology(SICT) are responsible for creating a Moodle course for each course as well as a Moodle course for each semester. 
Students are automatically assigned to the semester course.
The administrative personnel write some general information to the courses.
Additional information are written by the lecturer.
Course information is often copied and pasted different places.
