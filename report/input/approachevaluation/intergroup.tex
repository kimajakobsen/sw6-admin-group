\section{Inter Group Collaboration}
\label{sec:intergroup}
\todo{comment on change in our requirements}
In this section the chosen development and its utilization is discussed.

% Choice of scrum
We chose to use \sos{} as our development approach.
We chose this over XP and traditional based approaches.
It is difficult to validate that the choice is correct and that \sos{} is the best suited development approach compared to the others.
On the other hand; nothing indicates that they should be. 


In the implementation of \sos{} we chose to have \sos{} meetings nearly twice a week, which worked well since  they made all groups talk together at least once a week. 
In the beginning and prior to the employment of a development method we held large whole team assemblies. 
This practice worked well in the initial part of the project, but they became ineffective as progress went on. 
A reason for this is that the topics discussed early in the development is important and interesting knowledge to every one, but later some topics became of no interest to some group while important to others.
That made the enthusiasm at the big assemblies fall drasticly. 


An issue in the implementation of \sos{} is the lack of an on-site customer and a project owner. 
Both are described as key concepts. 
We compensated for these lacks by doing several interviews with possible end-users and to have them included in the demo meetings. 
The role of the project owner was done by our selves, which is not an optimal solution. 
It is not optimal for the reason that it is recognized as bad \scrum{} practice not to have a product owner to prioritize tasks \cite[P. 128]{larman}. 


Another issue is the lack of a shared room, which is preferred in \sos{} were not possible, but it helped that the group rooms are located close to each other and we quickly discovered that by letting the door stand open during work hours helped on the inter group communication. 

In general the development approach worked well and it made us adapt to changes. 
An example of a large change during development was when we realized that it were impossible to use activity modules in project groups. 
This was overcome by changing the already implemented activity modules into \block[]s. 
The changed requirement was embraced and quickly implemented. 