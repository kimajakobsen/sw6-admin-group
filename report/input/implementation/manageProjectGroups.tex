\section{Managing Project Groups} %Jeg skriver kun om admin funktionalitet her -Mikael ~♥
\label{sec:manProjGrpImpl}
Before any project group can be useful it has to be created.
\admpers[c] needs to have the ability to add, edit, and delete project groups.
%The administration tools we provide have to be easy and fast to use, since they potentially have to be used many times.
The features we provide to manage project groups are known as admin tools in Moodle.
This section describes how we implement the administration tools needed to manage project groups.


The main functionality is placed in the file \moodlefile{/admin/tool/projectgroup/lib- .php}, which is a code library consisting of approximately $500$ lines of code. 
The library is know henceforth as the admin tool library.
The library includes the project group library discussed in \secref{sec:pglib}, and hereby allowing pages in the administrative part to use code in the project group library. 
This inclusion is the implementation of the dependency between the admin component and the project group library component seen on \figref{fig:architecture}. 

In the following sections creation, editing, and listing of project groups are presented. 
Deletion of project groups is omitted, since we consider the functionality used as trivial. 
The reader can see the code for the deletion view in \moodlefile{/admin/tool/projectgroup/delete.php} and the function, \fu{delete\_projectgroup}, that does the actual deletion is located in the admin tool library.

\subsection{Add \& Edit}
\label{sub:addedditprojectgroups}
To add or edit project groups, the file \moodlefile{/local/projectgroup/edit.php} is loaded into the browser. 
The look and feel of this page is described in \secref{sec:adminPrensentation}.
The same file is used for both editing and adding of new project groups. 
The only difference between editing and adding a project group is that when editing a group an URL parameter, \vari{id}, is set.
When \vari{id} is set the form fields are filled with the relevant information from the database.
Once the submit button is pressed the function \fu{save\_or\_update\_projectgroup} is called with a project group object as input.
That function checks if the \vari{id} of the group is set. 
If a project group is being edited, a row in the database can be updated.
If not, a new project group is being created and a row is inserted in the database instead.

When \fu{save\_or\_update\_projectgroup} is used to update a project group the possible change in members is handled by removing every member from the project group and then adding the ones received as part of the arguments of the call.
This means that a project group briefly contains no members whenever it is updated, even when there is no change to the members or their roles.
The communication with the database is handled in a transaction, which means that the database system, MySQL, should ensure that the data is constantly in a consistent state.
The role of a member is as standard set to ``0'', which indicates a student. 
If a member is highlighted in the view when the form is submitted (as Kurt is in \figref{fig:moodlcgroupedit}) the role value is changed to supervisor (indicated by a value of ``1'') before the call to \fu{save\_or\_update\_projectgroup}.


\subsection{Lisiting}
In the list of project groups, mentioned in \secref{sec:adminPrensentation}, we provide a filtering functionality, which makes it possible to find a specific set of project groups.
The project group filtering is a rework of the user filtering, which already exists in Moodle.
The filter itself is placed in \moodlefile{/admin/tool/projectgroup/projectgroup\_filtering.php} and is a class named \cl{projectgroup\_filtering}.
This functionality is requested by the administrative personnel (see \secref{sec:requirements}).
The project group list is rendered in \moodlefile{/admin/tool/projectgroup/index.php} and the filter is created here as well. 
\coderef{moodlefiltering} shows the invocation of the filter. 
The filter is created and by the use of the function \fu{get\_projectgroup\_sel- ection\_data} we get the data, which we use in the creation of the projectgroup form, which handles the HTML rendering of the filter. 
\begin{lstlisting}[style=phpCode, caption=\myCaption{The invocation of the filtering mechanism}, label=moodlefiltering]
//create the user filter form
$filtering = new projectgroup_filtering();

//get the data from the form
$data = get_projectgroup_selection_data($filtering);
$data['fields'] = $filtering->get_fields();
$editform = new projectgroup_form(NULL,$data);
\end{lstlisting}\begin{comment}$\end{comment}

%\subsection{Deletion}
