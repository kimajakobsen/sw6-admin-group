
\section{Mikael Møller Hansen}
\label{sec:mikael}
Interviewers: Kim A. Jakobsen and Rasmus Veiergang Prentow\\

This interview was conducted on April 12\ths{} at Cassiopeia at Aablorg University.
At the time of writing Mikael is IT Administrator, he is engaged in developing a system that supports the daily work at the university

The following is a list of topics that summarize the interview.

\subsection*{AdmDB}
AdmDB is a tool used by administrative personnel at Cassiopeia.
It contains information about student and the project groups.
It was developed as a supportive system to the old Typo3 system. When project groups are typed into the system, a repository is created for the given group. A project group have the following properties:
\begin{itemize}
	\item A group room alias
	\item Room number
	\item Supervisor
	\item Students
	\item A period where the project group is active
	\item Repository
\end{itemize}

Mikael explains that AdmDB i necessary at the moment but he hopes that a central system will be developed to administrate project groups in all department on Aalborg University. 

\subsection*{\system{}}
We explain the concepts behind \system{} and asks Mikael to comment. 
We talk about using AdmDBs groups in \system{} thus allowing us to automatically create project groups. 
He replies that we should wait because he currently developing a new API for AdmDB using RESTful architecture. 

\subsection*{Other Systems}
Mikael mentions that AdmDB is not used on the entire university.
Other departments are using their own systems to administrate project groups, some of which are analog -- post-its on a wall.