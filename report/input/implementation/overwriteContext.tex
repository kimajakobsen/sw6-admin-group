\subsection{Context of Project Groups}
In \secref{sub:contextsystem}  the context system of Moodle is described.
In this section the creation of a custom context is described. 

To have different blocks for different virtual group rooms we need our own context.
We will create our own context level and class.
We call the context \cl{context\_projectgroup}. 
The class is located in \moodlefile{/local/projectgroup/context- .php} and is included by the project group library. 

Moodle does not support extension of contexts through any of the 34 different plugins types available. 
This fact poses a problem for us.
There are two parts of the problem.
The first is to create a new context and the second is to load it properly. 
We create a new context by making a new class, which is very similar to \cl{context\_course} (a class that is part of \moodle{} core code), and by defining the context level of project groups as a constant. 
The class header and the constant definition can be seen in \coderef{src:codeprojectgroupcontext}. 
The constant is set to 55, as seen in \lineref{src:codeprojectgroupcontext:level}.
It is chosen because that this context level is unused.
% and it is close to the course context level. 
%We regard the project group contexts to be at the same level as course contexts. 
%Intuitively project group concepts and courses are ranked along side each other.
%However, project groups does not have categories like courses does.

\begin{lstlisting}[style=phpCode, caption=\myCaption{The context\_projectgroup class header and constant definition}, label=src:codeprojectgroupcontext]
define ('CONTEXT_PROJECTGROUP',55); (*\label{src:codeprojectgroupcontext:level}*)
class context_projectgroup extends context {
...
\end{lstlisting}

When a context is loaded in a Moodle page it is instantiated by \fu{get\_context- \_instance}, which takes a context level and an instance id. 
The instance id can be a course id or similar depending on the context level -- in our case this would be a project group id. 
This function calls a static method in the \cl{context\_helper} class, which uses a private array to translate the context level into a class name.
As stated in \chapref{chap:systemDef} we refrain from changing the core code of Moodle. 
If this constraint were not enforced the array could simply be extended directly in the code.  
Since the array used is private we can not extend the context system by overriding the \cl{context\_helper} class. 

The newly created context is only used in pages created in this project and we can therefore create our own version of \fu{get\_context\_instance}. 
The new function can be seen in \coderef{src:codeprojectgroupcontextinstance}.
\begin{lstlisting}[style=phpCode, caption=\myCaption{The function to get projectgroup context}, label=src:codeprojectgroupcontextinstance]
function get_projectgroup_context_instance($instance = 0) 
{ 
    return context_projectgroup::instance($instance);
}
\end{lstlisting}

The \vari{instance} argument denotes a project group id.
%The \vari{strictness} variable indicate its compatibility mode.
%There are three constants defined in \moodle{} DML that are supposed to be used as input for \vari{strictness}.
%These are: \vari{IGNORE\_MISSING}, \vari{IGNORE\_MULTIPLE}, and \vari{MUST\_EXIST}.
%The value of \vari{strictness} is passed on in the static method \me{instance} of \cl{context\_projectgroup} to the \moodle{} database API call to \fu{get\_record} which implements functionality to throw exceptions depending on \vari{strictness}.

With the new context and the function to instantiate it we can now add blocks to specific virtual group rooms. 