\section{\system{}}
\label{sec:productsystem}
\system{} is the system we as the four groups collaboratively have developed.
In this section we will evaluate on the the product, which we created.


In this project we created an extension to \moodle{}, which supports education through the Aalborg PBL model. 
The extension, consisting of several smaller extensions, gives the students and their supervisor a virtual representation of their project group. 
The \detdeandrelaver{}s allow students to communicate internally through a message board and a black board, and communicate with their supervisor through the supervisor message board.
%These boards makes it easy to save communication threads and thereby not loosing them in email conversations. 
The students can make appointments with themselves and/or their supervisor through the timeline tool.
The virtual group room is a place where all the \detdeandrelaver{}s are available.
Each project group has a virtual group room associated to it.


\subsection{Decomposition}
We chose to split the entire multi-project into four parts based on the core principles of PBL. 
In retrospect the split could have been done more harmonic than it is.
Our group got the task of implementing the project group administration and the virtual group room, which is two very separate tasks with different end user groups. 
The black board tool could be merged with the message board tool by integration an existing drawing application, such as google apps, which is already integrated in Moodle by a 3\rd{} party extension~\cite{moodlegoogleapp}. 
A new decomposition will then be:
\begin{itemize}
	\item Management
	\item Planning
	\item Communication
	\item Virtual Group Room
\end{itemize}

This division would have give a more distinct group of end users to all components.
The distribution of end users would be:
\admpers{} to Management, students to the others, where Communication would have to consider supervisors as well.



\subsection{Architecture}
% Architecture
%The chosen architecture used in the implementation of \system{} works as intended and the different \subgroup{}s managed to implement their component so they comply with the architecture. 
%The architecture helps illustrate the relations between the different components of the system, which is helpful for the groups who are to continue the work. 



The architecture was not the result of an up front analysis but rather the result of a few iterations of development.
We discovered that there was a need for a specification of where to put new functionality.
In particular we wanted a shared code library where functionality that was used be some or all parts of \system{} could be placed.
This resulted in our three-layered architecture (not counting the database and \moodle{} core) described in \secref{sec:architecture}.
It has allowed us to have a common understanding of how we wanted the final system to be.
It has also given guidelines regarding where to put functionality and has encapsulated related functionalities.
This should help the developers that will take over this project next year to add new features as well as changing existing features.



\subsection{Database}
%Database
The database schema used in this project is fairly simple.
It only consists of two tables, namely  \rel{projectgroup} and \rel{projectgroup\_members}. 
These two tables represent the entity project group and the relation that exists between project groups and users in \moodle{}. 

The \field{longname} field in \rel{projectgroup} is by Lene W. Even considered useless, as described in \appref{sec:lenedemotwo}. 
The field is optional, but it should be removed if members of the \admpers{} from other departments confirm that there is no need for it. 
In the \rel{projectgroup\_members} there are two time fields -- \field{created} and \field{updated}. 
These fields are unused in the current implementation. 
Since the members of a project group are all deleted whenever a project group is updated these fields are worthless.
For these fields to have a purpose the implementation must be changed to update entries in \rel{projectgroup\_members} and set \field{updated} appropriately without modifying \field{created}.

%The field \field{created} can be useful if at some point it is necessary to know when a user becomes a member of a group, while \field{updated} is not used since the application never updates a row, but removes and inserts it again if changes are applied.

