\section{subsystem}
In this section the subsystem implemented by our group is evaluated. 
The subsystem created can be split into two parts, the management part and the project group room and are implemented differently. 
%\subsection{

% Architecture


% Project group
We decided implement the concept of project groups in Moodle to achieve our goal of giving Moodle a better fundament for PBL\todo{rewrite}.

%% Team vs. Shared
In section XXX we decide to implement the project group room on a peer-group basis, which gives a group its own room for collaboration. 
The shared room is infeasible, as argued in that section, but the concept of projects can be implemented. 
Project are as described in section XXX an entity which is per semester. 
A project can have a page in the same way a project group has a page, which will allow for collaboration and communication between groups in a semester. 


%%% Permissions

%%% Context
In the implementation of the project group room we created our own context,  which gives a better solution compared to linking project groups to courses. 
such a link would create an unnatural relation, which does not reflect the problem domain.   

%% Container (Template requirement)
%% Navigation






% Administration
%%


\subsection{Administration}
We decided that management of project groups should only be done by administrative personnel. 
This is not a final decision and we recon that allowing users to create their own project groups is not a bad idea. 
With student management of project groups there are several constraints which must be considered, they are discussed in section XXX. 
The resulting system and its usage thereof will be slightly different. 
When the administrative personnel manages project groups the system has a possibility to become authority. 
That is, the system can be used as primary project group database. 
To become authoritative project group database the structure of the project groups and projects which are discussed in section XXX must be changed to be more general to support all structeres at the university(MIKAEL) and the number of Moodle installations at AAU must be reduced to one. 
With student management the authority of the system is removed and will only be a tool for students and their supervisors.

%In section \secref{sec:groupManagement} we decided to implement the project group management as an admin tool in Moodle and thereby only allowing people with administrative privileges to manage project groups. 





%Database
The database schema used in this project is fairly simple, hence it only consist of two tables, namely  \rel{preojectgroup} and \rel{projectgroup\_members}. 
These two represents the entity project group and the relation that exists between project groups and users in Moodle. 

The \field{longname} field in \rel{projectgroup} is by Lene W. Even considered useless, see app XXX. 
The field is optional, but it should be removed if no body will use it. 
In the \rel{projectgroup\_members} there are two time fields named, \field{created} and \field{updated}

% Testing








