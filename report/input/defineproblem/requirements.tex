\section{Requirements}
\label{sec:requirements}
During this project we have had several meetings with personnel that we consider to be our end users.
Some of them before coding the system and some of them during coding.
In this section we present our the final requirements.
The continues change in our requirements is evaluated in \secref{sec:intergroup} in \partref{part:evaluation}.
The interviews that have been conducted to gather these requirements are found in \appref{app:interview}.

Since we use scrum as our main development practice we use a product backlog to capture our requirements~\cite[p.~114]{Larman04}.
Different sources suggest different notations for what is in a backlog~\cite[p.~17]{scrumchecklist}\cite[pp.~123-124]{Larman04}.
Our backlog items can be a use cases, features, or fixes.
Since this project is to be proceeded by a new group of students next year, we allow our product backlog to be larger than we are able to complete in this project.
We will now focus on the items that are part of this semester project.
These are said to be part of our release backlog.
The backlog items that makes up our product backlog for our \subsystem{} are defined in \figref{fig:productbacklog}.
%The process' behind creation of the most central backlog items are also presented here.
We will now present three different examples of how we derived our backlog items.

\paragraph{Virtual Meeting Place}
The concept of a virtual meeting place comes from the meeting with ELSA described in \secref{sub:elsaInterview}. 
From the interview with Lene in \appref{sec:lene} we know that she manually types a list of groups and their members into Moodle.
Moodle is not able to interpret this. 
In the interview with Jette and Pia (\appref{sec:jettepia}) we learn that a member of the \admpers{} is responsible for the formed groups. 
This implies that there exist records of the composition of the groups.   
In contrast to the these two interviews then Morten explains that on the MPBL education the \admpers{} do not manage the project groups.
Based on this information we decide that we will make a virtual meeting place that should reflect the project groups that exist on Aalborg university. 
We decide this knowing that it will cause additional work for the \admpers{} on the MPBL education. 
We will elaborate on the virtual meeting place in \chapref{chap:analdesign}.

\paragraph{Find Project Groups}
We deem that it is necessary to create a list with all the project groups that exist.
This should be available to the part of our target group that manage the project groups.
As mentioned in \secref{sec:developmentpractice} this project is conducted agile. 
In a demo meeting, in our third sprint, with Lene we discovered that it was difficult to find a given project group where there exist several in the system.
The demo meeting can be found in \appref{sec:lenedemoone}. 
We decide to make functionality that will allow users to search in the list of project groups.
The implementation of this concrete problem will be addressed in \secref{sec:manProjGrpImpl}.

\paragraph{Manage Project Groups}
It should be possible to manage the project groups. In our considerations of how this should be designed, we consider how the creation of courses are done. 
ELSA are responsible for creating new courses at the start of every semester. 
This is done by the \admpers{} from every institute informs ELSA of which courses that should be created. 
ELSA creates the courses based on a general templates.
The \admpers{} and lectures then populates the courses with data. 
Information about this process can be read from \appref{sec:lene}, \secref{sub:elsaInterview},  \appref{sec:jettePia}, and \appref{sec:thomas}.
Similar to the creation of courses should it be possible to create and in other ways to manage project groups. 
The design of project groups will be explained in \secref{}. \\

We will not explain all the backlog items because we reckon that the derivation of them are of similar nature.
Now we will account for the choice of the release backlog.

\subsection{Release Backlog}

 %use these requirements to make our system statement.





\begin{comment}
%skriv flere cases af hvordan b-items er opstået 
In the interview described in \appref{sec:lene} we learned that there is a need for having a concept of project groups in \moodle{}.  to enhance communication by sending messages from the \admpers{} to project groups through \moodle{}.
This evolves into two backlog items; integrate the concept of project groups into \moodle{} and allow the administrative personnel to post messages to the project group.
The former is represented as a backlog item which can be seen in \figref{fig:productbacklog}.
The latter is implemented by the \supervisorgroup{}.
\end{comment}


\begin{figure}%
\begin{tabular}{|p{0.2\textwidth}|p{0.1\textwidth}|p{0.7\textwidth}|}
	\hline
	
	\textbf{Name} & \textbf{Priority} & \textbf{Description} \\
	\hline
	Manage Project Groups & Priority & A subset of our target group should be able to manage the project groups, this include create, edit, and delete project groups.  \\
	\hline
	Virtual Meeting Place & Priority & Create a virtual space that are compliant with the principals of the Aalborg PBL model and create a feeling of nearness.  \\
	\hline
	\textit{Recursive Project Groups} & Priority & \textit{Allow project groups to contain groups of persons. This also include the administrative tools to manage the groups and sub-groups.} \\
	\hline
	Introduce Roles & Priority & Add the possibility to assign roles to the members of a project group, roles such as supervisor and student.   \\
	\hline
	\textit{Manage Roles} & Priority & \textit{Create Functionality to define and manages roles. An example of a new role  could be secondary supervisor} \\
	\hline
	Find Project Groups & Priority & Produce functionality to find a given project group. This involves creating a list of project groups and making features to allow search or filtering. \\
	\hline
	Navigate to Project Groups & Priority & The members of the project group should have a way to navigate to it. \\
	\hline
	Project Group Members & Priority & It should be possible to see who the members of the group is, to create the feeling of nearness. \\
	\hline
	\textit{Project Group Synchronization} & Priority & \textit{Project groups should reflect the physical project groups at the university. Thus when a student joins an existing group, the student should automatically be added to the virtual group room.}  \\
	\hline 
	\textit{Virtual Meeting Place Template} & Priority & \textit{To account for the user groups diversity and different needs, a set of templates, that defines the visual appearance of the virtual meeting place, could be applied when creating a new virtual meeting space }  \\
	\hline 


	
	
	
\end{tabular}
\morscaption{A list showing features from our product backlog. The italic items are not part of our release backlog}
\label{fig:productbacklog}
\end{figure}

\todo{Skriv evt noget om hvorfor vi har valgt at lave dem vi har lavet.}
\FloatBarrier