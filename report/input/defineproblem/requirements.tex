\section{Requirements}
\label{sec:requirements}
During this project we have several meetings with end user representatives.
We hold meetings before, during, and after the development of the system. 
We do this to gather the requirements of the end users based on the system under development.
In this section we present our final requirements.
The continuous change in our requirements is evaluated in \chapref{chap:evalDevApproach} in \partref{part:evaluation}.
The interviews that have been conducted to gather these requirements are found in \appref{app:interview}.

Since we use \scrum{} as our development method we use a product backlog to capture our requirements~\cite[p.~114]{Larman04}.
%Different sources suggest different notations for what is in a backlog~\cite[p.~17]{scrumchecklist}\cite[pp.~123-124]{Larman04}.
%Our backlog items are all use cases, features, or fixes.
Since this multi-project is to be proceeded by a new multi-project group next year, we allow our product backlog to be larger than we are able to complete in this semester.
In this section we focus on the items that are part the project of this semester.
These are said to be part of our release backlog.
The backlog items that make up our product backlog for our \subsystem{} are defined in \figref{fig:productbacklog}.
%The process' behind creation of the most central backlog items are also presented here.
The following three paragraphs will describe how we derived some of our backlog items.

\begin{figure}%
\begin{subfigure}[b]{\textwidth}
\begin{tabular}{|p{0.2\textwidth}|p{0.8\textwidth}|}
	\hline
	\textbf{Name} & \textbf{Description} \\
	\hline
	Manage Project Groups & A subset of the end users should be able to manage the project groups, this includes creating, editing, and deleting project groups.  \\
	\hline
	Virtual Meeting Place & Create a virtual space that is compliant with the principles of the Aalborg PBL model and create a feeling of nearness.  \\
	\hline
	Introduce Roles & Add the possibility to assign roles to the members of a project group. \\
	\hline
	Find Project Groups & Produce functionality to find a given project group.\\
	\hline
	Navigate to Virtual Group Room & The users of project groups must be able to navigate to the their virtual group room. \\
	\hline
	Presentation of Members & It should be possible to see who the members of the group is, to create the feeling of nearness. \\
	\hline
	\end{tabular}\\
	\morscaption{Items in the release backlog}
\end{subfigure}
	
	\vspace{10 mm}
	\begin{subfigure}[b]{\textwidth}
	\begin{tabular}{|p{0.2\textwidth}|p{0.8\textwidth}|}
	\hline
	\textbf{Name} & \textbf{Description} \\
	\hline
	Archiving Project Group & {ELSA has requested that data archiving should be possible.} \\
	\hline 
	Recursive Project Groups & {Allow project groups to contain other project groups. This also includes the administrative tools to manage the new structure of project groups.} \\
	\hline
	{Manage Roles} & {Create Functionality to define and manages roles. An example of a new role  could be secondary supervisor.} \\
	\hline
	{Project Group Synchronization} & {Project groups should be added automatically by synchronizing the system with a central database containing all the project groups of the university.}  \\
	\hline 
	Virtual Meeting Place Template & {To account for the end users diversity and different needs, a set of templates that defines the visual appearance of the virtual meeting place could be applied when creating a new virtual meeting space.}  \\
	\hline
\end{tabular}
\morscaption{Items not in the release backlog}
\end{subfigure}
\morscaption{Lists showing features from our product backlog}
\label{fig:productbacklog}
\end{figure}

\paragraph{Virtual Meeting Place}
The concept of a virtual meeting place comes from the meeting with people from the MPBL education described in \secref{sub:mpblInterview}. 
\begin{comment}
From the interview with Lene in \appref{sec:lene} we know that she manually types a list of groups and their members into Moodle.
%Moodle is not able to interpret this.
In the interview with Jette and Pia (\appref{sec:jettePia}) we learn that a member of the \admpers{} is responsible for the formed groups. 
This implies that there exist records of the composition of the groups.   
In contrast to the these two interviews Morten explains in an interview (\appref{sec:morten}) that on the MPBL education the \admpers{} do not manage the project groups.

Based on this information we decide that we will make a virtual meeting place that reflects the project groups that exist on Aalborg university. 
We decide this knowing that it will cause additional work for the \admpers{} on the MPBL education. 
We will elaborate on the virtual meeting place in \chapref{chap:analdesign}.
\end{comment}
It is supposed to be a virtual place in which members of a project group can meet and conduct project related work.
Additionally, it must create a feeling of nearness as mentioned in our discussion with people from the MPBL education.


\paragraph{Find Project Groups}
\begin{comment}
We deem that it is necessary to create a list with all the project groups that exist.
This should be available to the part of our target group that manage the project groups.
In our third sprint we held a demo meeting where Lene mentioned that it was difficult to find a project group when there were many project groups in the system(see \appref{sec:lenedemoone}).
We decide to make functionality that will allow users to search in the list of project groups.
The implementation of this concrete problem will be addressed in \secref{sec:manProjGrpImpl}.
\end{comment}
For the managers of project groups there must be functionality available to find a specific project group or a set of specific project groups.
This requirement is based on a demo meeting that we conducted with Lene (see \appref{sec:lenedemoone}).


\paragraph{Manage Project Groups}
It should be possible to manage the project groups. 
We let us inspire by how courses are currently created at Aalborg University.
ELSA is responsible for creating new courses at the start of every semester. 
This is done by the \admpers{}.
Every department informs ELSA of which courses that should be created. 
ELSA creates the courses based on a general template.
The \admpers{} and lecturers then populate the courses with data. 
Information about this process can be read in \appref{sec:thomas}, \appref{sec:lene}, \appref{sec:jettePia}, and \secref{sub:elsaInterview}.
Similar to the creation of courses it should be possible to create and in other ways manage project groups. 
The design of project groups are explained in \chapref{chap:analdesign}. \\

Not all backlog items are explained because we believe that the derivation of them are of similar nature.
In the following section the choices of the release backlog are accounted for.

\subsection{Requirements Selection}
\label{sec:releaseBacklog}
Recall the table seen in \figref{fig:productbacklog}. 
To explain the selection of requirements we present a short description of and an argumentation for why other requirements are excluded.
The four items are described below.

\paragraph{Recursive Project Groups}
The idea of allowing recursive groups is that multi-projects such as ours are better structurally supported.
In an interview with Lene (see \appref{sec:lenedemoone}), we discussed the possibility of having a recursive project group structure.
Lene could not imagine any concrete cases where such a structure would be beneficial in practice, which is why we felt that it is not a feature that will improve the system substantially. 

\paragraph{Project Group Synchronization}
In the Department of Computer Science at Aalborg University AdmDB (Administration Database) is, among other things, used to manage project groups. 
During the interview with Mikael (see \appref{sec:mikael}) we learned that he is currently developing a new Application Programming Interface (API) for the AdmDB system, and it would not be optimal to rely on the old API.
We also learned that there does not exist a university-wide system that contains all project groups. 
We believe that if this feature should be implemented it should be after the new API is finished.
We elaborate further on this subject in \secref{sub:automanagement}. 

\paragraph{Virtual Meeting Place Template}
As previously mentioned ELSA is responsible for creating course pages in Moodle. 
They do this using a template. 
It is desirable to have the same functionality when creating the virtual meeting places, but this feature will not bring any immediate value to the product. 
We postpone this feature until there arises a concrete need for such functionality.

\paragraph{Archiving Project Groups}
\label{sub:analysarchiving}
% Archiving (ELSA, Lene & Jette) (Indbygget archive restore og courses)
From the meeting with ELSA (see \secref{sub:elsaInterview}) we know they archive courses at the end of each semester. 
Lene explains that it is to preserve the content for future reference (see \appref{sec:lene}). 
Morten (see \appref{sec:morten}) explains that teachers use the archived versions of a course to create a new course of the same subject. 
%We considered that project groups can be archived for the same purpose so supervisors and students can look back at their previous work.
Archiving is worth considering because it allows students and supervisors to look back at their previous work.
%In Moodle it is possible to backup and restore courses, which allows teachers to take a snapshot of their course materials. 
%This feature is not used by ELSA to archive courses.
%Instead they copy the courses and its contents.
%A similar approach can be used to archive project groups for future reference. 

We believe that the archiving of project groups is a very beneficial feature for students, supervisors, and administrative personnel.
However, since we have limited development time we suggest that future student projects could look into this feature.
A further discussion on archiving can be found in \chapref{chap:futurework}.

 %use these requirements to make our system statement.

\subsection{Non-functional Requirements}
To capture our non-functional requirements, also referred to as the ``-ilities''~\cite[p.288]{Larman04}, we rely on our end users' experiences with the system.
Common non-functional requirements include availability, robustness, effectiveness, efficiency, extensibility, and usability~\cite[sec.~9.1]{roedeaalborg}.
We primarily focus on robustness, extensibility, and usability.
The reason for the two first is that new developers will take over the project next year. 
They may want to change something, which is why we must ensure that the software is robust. 
They probably want to extend our system with new features which calls for an extendable system.
We believe usability is very important, since the system will potentially be used by many students with varying experience with LMSs.
Availability is not considered since our system is an extension to another system and our availability is directly connected to that of the main system, \moodle{}.



\FloatBarrier
