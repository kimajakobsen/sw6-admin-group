\section{Morten Mathiasen Andersen Interview}
\label{sec:morten}
Interviewers: Kim A. Jakobsen and Alex Bondo Andersen\\

This interview was conducted on March 22\ths{} at Department of Development and Planning at Aablorg University.
Morten is working at MPBL. 
MPBL educate students in Problem based learning.
All education is conducted online and the students are often residents in foreign countries. 
At the time of writing Morten is an Assistant at Aalborg University, and supports lectures and students in daily work with Moodle.

The following is a list of topics that summarize interview.

\subsection*{Courses}
At the start of a new semester, ELSA creates new Moodle course pages, Morten -- in cooperation with the course lecturer -- populates the course page. 
In this process Morten uses archived course material when filling out the course page.

\subsection*{Projects}
When projects are conducted the students are responsible for forming groups. 
The formed groups are not administered by the university.
Morten explains that Skype and or Adobe Connect is used as communication medium, this is necessary because of geographical differences. 
A common problem for the group are how to share files, thus are all groups using different tools. 
If all tools where integrated in Moodle, it would make the administration task easier.

\subsection*{Improvements to Moodle}
All information is provided through Moodle forums. 
To insure that everybody reads the information, a system could be developed that alerts the user of unread posts.
Moodle course pages are static in the appearance, Morten wishes that it was be possible to add banners to the individual course page.
Lectures at MPBL are held online via video steaming tools. 
It is desirable to chat through Moodle during the lectures.   
